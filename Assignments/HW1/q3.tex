\section*{سوال ۳}

ابتدای فصل اول کتاب پرسمن، داستانی از حضور آقای پرسمن در یک شرکت بازیسازی آورده شده است. این داستان را بخوانید. به نظرتان چرا آقای پرسمن این داستان را در ابتدای کتاب خود آورده است؟

\section*{جواب سوال ۳}

در یک شرکت بازی‌سازی است. این داستان به خواننده نشان می‌دهد که آقای پرسمن در عمل به مشکلات و چالش‌های مربوط به توسعه نرم‌افزار برخورد کرده است و تجربه‌های خود را با خواننده به اشتراک می‌گذارد.

آقای پرسمن این داستان را در ابتدای کتاب آورده است تا خواننده را با مفاهیم و مسائل مربوط به مهندسی نرم‌افزار آشنا کند. این داستان می‌تواند به خواننده نشان دهد که توسعه نرم‌افزار چقدر پیچیده و چالش‌برانگیز است و نیاز به روش‌ها و رویکردهای مناسب دارد. همچنین، این داستان می‌تواند از خواننده خواسته‌هایی مانند تفکر سیستمی، تحلیل و طراحی نرم‌افزار، مدیریت پروژه و کیفیت نرم‌افزار را درک کند.

به طور کلی، این داستان می‌تواند به خواننده کمک کند تا با مفاهیم اساسی مهندسی نرم‌افزار آشنا شود و درک بهتری از محتوای کتاب پیدا کند.

برای بررسی جزئی‌تر دلیل استفاده از این داستان در ابتدای کتاب به تحلیل جزئیات آن می‌پردازیم:

اهمیت مهندسی نرم‌افزار: این داستان نشان می‌دهد که حتی در صنایعی که شاید به ظاهر نیازی به رویکردهای سختگیرانه‌ی مهندسی نرم‌افزار ندارند، مانند صنعت بازی‌های ویدئویی، استفاده از تکنیک‌ها و اصول مهندسی نرم‌افزار ضروری است.

تطبیق‌پذیری مهندسی نرم‌افزار: داستان بر لزوم انطباق و تطبیق اصول مهندسی نرم‌افزار با نیازهای خاص یک شرکت یا یک پروژه تأکید دارد، نشان داده می‌شود که چگونه می‌توان اصول کلی را برای رسیدن به اهداف خاص، شخصی‌سازی کرد.

پیچیدگی‌های فزاینده در توسعه نرم‌افزار: با افزایش پیچیدگی‌های بازی‌ها و کوتاه‌ شدن چرخه‌های توسعه، نیاز به رویکردهای منضبط‌تر در توسعه نرم‌افزار بیشتر شده است. این داستان بر اهمیت استفاده از رویکردهای مهندسی نرم‌افزار برای موفقیت در چنین محیط‌هایی تأکید می‌کند.

تأکید بر نقش «کریتیوها»: داستان به اهمیت تعامل بین تیم‌های خلاق و توسعه‌دهندگان فنی اشاره می‌کند و اینکه چگونه نیازهای خلاقانه باید به زبان فنی ترجمه شوند تا امکان‌پذیر شدن توسعه واقعی فراهم آید.

در نهایت، با قرار دادن این داستان در ابتدای کتاب، آقای پرسمن سعی دارد ارتباط عمیقی بین نظریه مهندسی نرم‌افزار و کاربرد عملی آن در صنعت ایجاد کند، و ما را تشویق کند که به این ارتباط اهمیت دهیم و از مفاهیم کتاب برای حل مسائل واقعی استفاده کنیم.
