\section*{سوال ۵}

	آیا «متدولوژی ایجاد نرم‌افزار» همان «فرآیند ایجاد نرم‌افزار» است؟ از پاسخ خود با جزئیات و تفصیل دفاع کنید.
	
	در «مدل فرآیند عمومی ایجاد نرم‌افزار» در کتاب پرسمن، پنج «فعالیت چارچوبی» معرفی می‌شود که یکی از آن‌ها مدل‌سازی است.
	
	در این فعالیت به صورت کلی چه اقداماتی انجام می‌شود؟ امروزه در عمل چه زبان مدل‌سازی استاندارد شده است؟
	
	توضیح دهید که خروجی‌های فعالیت مدل‌سازی، چگونه و با چه هدفی در فعالیت بعدی از فرآیند عمومی ایجاد نرم‌افزار مورد استفاده قرار می‌گیرد.
	
	آیا فعالیت مدل‌سازی با اصل ششم از اصول دوازده‌گانه‌ی چابک در تناقض است؟ از پاسخ خود با مثال دفاع کنید.


\section*{جواب سوال ۵}

\textbf{متدولوژی ایجاد نرم‌افزار:} مجموعه‌ای از رویه‌ها، تکنیک‌ها، ابزارها و رویکردهای استاندارد است که برای تولید نرم‌افزار استفاده می‌شود. این رویکردها می‌توانند از مراحل تحقیق و توسعه تا تست و نگهداری نرم‌افزار کشیده شود.

\textbf{فرآیند ایجاد نرم‌افزار:} مراحلی است که یک نرم‌افزار از ابتدای توسعه تا انتشار و نگهداری طی می‌کند. هر متدولوژی ممکن است یک فرآیند خاص را برای توسعه نرم‌افزار تعریف کند.

پس، فرآیند ایجاد نرم‌افزار یکی از جزء‌های متدولوژی است.

فرآیند ایجاد نرم‌افزار و متدولوژی ایجاد نرم‌افزار اگرچه به نظر می‌رسد از نظر مفهومی به یکدیگر نزدیک هستند و گاهی اوقات به جای یکدیگر استفاده می‌شوند، اما تفاوت‌های اساسی با یکدیگر دارند.

\subsection*{تفاوت‌ها}

\begin{itemize}
	\item تمرکز فرآیندی نسبت به چارچوب متدولوژیک:
	
	فرآیند ایجاد نرم‌افزار ممکن است شامل مجموعه‌ای از قدم‌های مشخص برای توسعه نرم‌افزار باشد، مانند نیازسنجی، طراحی، پیاده‌سازی، آزمایش و تحویل.
	
	متدولوژی ایجاد نرم‌افزار، از سوی دیگر، شامل فرآیند مذکور به همراه فلسفه‌ها، ابزارها، روش‌ها و بهترین شیوه‌هایی است که چگونگی اجرای هر کدام از این قدم‌ها را تعیین می‌کند. به عنوان مثال، متدولوژی چابک تاکید بر توسعه تدریجی، همکاری نزدیک با مشتری و تطبیق‌پذیری دارد.
	
	\item ابزارها و تکنیک‌ها:
	
	یک فرآیند ممکن است استفاده از تکنیک‌های خاصی مانند UML برای طراحی یا JUnit برای آزمایش را پیشنهاد دهد.
	
	یک متدولوژی ممکن است فراتر از تکنیک‌های مشخص برای طراحی و آزمایش رفته و فرهنگ سازمانی، نگرش‌ها، ارزش‌ها و اصولی را که باید در تمامی جنبه‌های توسعه نرم‌افزار رعایت شوند، معرفی کند.
	
	\item مقیاس و دامنه:
	
	فرآیندها معمولاً کوچکتر و بیشتر به جنبه‌های عملیاتی توسعه نرم‌افزار مربوط می‌شوند.
	
	متدولوژی‌ها ممکن است در یک دامنه وسیع‌تر و با دیدگاهی جامع‌تر به مدیریت پروژه، استراتژی‌های ارتباطی، آموزش و توسعه تیم، و ملاحظات استراتژیک کلان نگاه کنند.
\end{itemize}

\subsection*{مثال عینی: Scrum به عنوان متدولوژی}

فرآیند در Scrum ممکن است به سری از اسپرینت‌های دو هفته‌ای اشاره کند که در هر یک اهداف کوتاه‌مدت تعیین و دنبال می‌شوند.

متدولوژی Scrum ، این فرآیند را در چارچوبی از قواعد، نقش‌ها (مانند
\lr{Scrum Master}
و 
\lr{Product Owner})، جلسات (مانند دیلی استنداپ) و ابزارها (مانند کانبان برد) قرار می‌دهد و این‌ها همگی به همراه مجموعه‌ای از ارزش‌های کلیدی مانند اعتماد، شفافیت و تعهد مطرح می‌شوند.

\section*{مدل‌سازی}

مدل‌سازی در زمینه توسعه نرم‌افزار، فرایندی است برای ایجاد یک مدل مفهومی از یک سیستم کامپیوتری که شامل نرم‌افزار و گاهی اوقات سخت‌افزار مرتبط با آن می‌باشد. مدل‌ها می‌توانند ساختار، رفتار و نحوه تعامل اجزای سیستم با یکدیگر و با کاربران را نشان دهند. در فرایند مدل‌سازی، معمولاً اقدامات زیر صورت می‌گیرد:

\begin{itemize}
	\item تعریف نیازمندی‌ها: درک و تعریف دقیق نیازمندی‌های سیستم که باید توسط مدل پوشش داده شوند.
	\item انتخاب روش مدل‌سازی: تعیین اینکه از چه نوع مدل‌سازی استفاده شود، مانند مدل‌های انتزاعی، مدل‌های داده، مدل‌های رفتاری، یا مدل‌های تعاملی.
	\item تعریف مفاهیم: مشخص کردن مفاهیم کلیدی و موجودیت‌های مورد نیاز برای مدل، مانند کلاس‌ها، شی‌ءها، عملیات، فرایندها و تعاملات.
	\item طراحی مدل: استفاده از ابزارها و نمادهای استاندارد برای ترسیم مدل، مثل نمودارهای UML یا BPMN
	\item تحقق و اعتبارسنجی: ساختن یک نسخه اولیه از مدل و اعتبارسنجی آن برای اطمینان از دقت و کارایی در نمایش مفاهیم و روابط واقعی.
	\item تکرار و بهبود: بازبینی و بهبود مدل بر اساس بازخورد و یافته‌های جدید برای اطمینان از دقت و کامل بودن مدل.
	\item مستندسازی: تهیه مستندات کامل و دقیق از مدل و توضیحاتی در مورد چگونگی تعامل اجزاء مختلف.
	\item تست و شبیه‌سازی: استفاده از مدل برای تست سناریوهای مختلف و شبیه‌سازی رفتار سیستم قبل از پیاده‌سازی واقعی.
	\item انتقال مدل به طراحی: تبدیل مدل‌های تایید شده به معماری‌ها و طراحی‌هایی که می‌توان بر اساس آن‌ها نرم‌افزار را پیاده‌سازی کرد.
\end{itemize}

فرایند مدل‌سازی به توسعه‌دهندگان کمک می‌کند تا یک درک مشترک از سیستم و نیازمندی‌های آن پیدا کنند و همچنین به انتقال دانش در میان تیم و ذینفعان کمک می‌کند. همچنین این فرایند می‌تواند در کاهش خطاها و سوءتفاهمات در مراحل بعدی توسعه نرم‌افزار مفید باشد.


\section*{فعالیت مدل‌سازی در فرآیند عمومی ایجاد نرم‌افزار}

در فعالیت مدل‌سازی، نیازها و مشخصات پروژه به صورت دقیق و واضح مدل‌سازی می‌شوند. این مدل‌ها می‌توانند شامل نمودارهای کلاس، نمودارهای فرآیند، و نمودارهای توالی باشند. زبان مدل‌سازی استاندارد شده امروزه UML یا
\lr{Unified Modeling Language}
است. این زبان، یکی از زبان‌های استاندارد برای مدل‌سازی نرم‌افزار است که برای توصیف و طراحی سیستم‌ها استفاده می‌شود.

خروجی‌های فعالیت مدل‌سازی، به عنوان نقشه راه برای برنامه‌نویسان و توسعه‌دهندگان عمل می‌کنند و به آن‌ها کمک می‌کند تا با دیدی روشن‌تر، به طراحی و پیاده‌سازی سیستم بپردازند.

\section*{تناقض مدل‌سازی با اصول چابک}

متدولوژی‌های نسل سومی، مانند
\lr{Rational Unified Process (RUP)}
بر فرایندهای ساختاریافته و مرحله‌ای تأکید زیادی داشتند و به طور معمول شامل مراحل مشخصی بودند که باید به ترتیب دنبال می‌شدند. در این متدولوژی‌ها، مدل‌سازی نقش محوری داشت و اغلب با استفاده از ابزارهای مهندسی نرم‌افزار و نمودارهایی مانند UML انجام می‌شد. هدف از مدل‌سازی، ایجاد یک نمایش دقیق و کامل از سیستم قبل از آغاز برنامه‌نویسی واقعی بود، به طوری که تمام جنبه‌ها و پیچیدگی‌های سیستم در مدل‌ها در نظر گرفته شده باشد.

\subsection*{\lr{Rational Unified Process (RUP)}}
RUP
به عنوان یک متدولوژی تکراری و تدریجی، به توسعه دهندگان این امکان را می‌داد که بخش‌های مختلفی از نرم‌افزار را در مراحل مختلف توسعه بسازند و بهبود ببخشند، که هر کدام ممکن بود شامل مدل‌سازی باشد. در RUP ، مدل‌سازی به عنوان یک ابزار برای کاهش ابهامات، پیش‌بینی مشکلات احتمالی و تسهیل ارتباط بین اعضای تیم در نظر گرفته می‌شد.

\subsection*{ظهور رویکردهای Agile}
با ظهور رویکردهای Agile ، تمرکز از مدل‌سازی و مستندسازی گسترده به سمت توسعه سریع و انعطاف‌پذیر منتقل شد. اصول Agile به کارآیی و سادگی تأکید دارند، و اصل ششم منشور Agile بیان می‌کند که روش ارتباطی مؤثرترین و کارآمدترین روش انتقال اطلاعات به تیم توسعه و درون آن است مکالمه رو در رو است. این موضوع به بحث و مناقشه‌ای در جامعه توسعه نرم‌افزار منجر شد، که آیا مدل‌سازی به طور کامل از بین خواهد رفت یا خیر.

\subsection*{تعادل بین مدل‌سازی و گفتگوی رو در رو}
در واقعیت، Agile نفی کامل مدل‌سازی را مد نظر ندارد، بلکه به دنبال یافتن تعادل مناسب بین مدل‌سازی و ارتباطات غنی است. در این رویکرد، مدل‌سازی هنوز هم می‌تواند به عنوان ابزاری برای فکر کردن از طریق مشکلات و ارتباط بصری مفاهیم پیچیده به کار رود، اما با حجم کمتر و بیشتر به عنوان ابزاری برای پشتیبانی از گفتگوی رو در رو استفاده می‌شود، تا اینکه به عنوان یک اسناد نهایی در نظر گرفته شود.

توسعه نرم‌افزار در طول زمان تکامل یافته و همچنان در حال تغییر است. اگرچه مدل‌سازی و مستندسازی دقیق در متدولوژی‌های نسل سومی نقش بزرگی داشتند، اما با پیشرفت رویکردهای Agile و تمرکز بر انعطاف‌پذیری و سرعت، جایگاه این تکنیک‌ها تغییر کرده است. در حالی که رویکردهای Agile مدل‌سازی را به کلی رد نمی‌کنند، آن‌ها استفاده از مدل‌ها را به شیوه‌ای متفاوت توصیه می‌کنند، به گونه‌ای که پشتیبانی‌کننده ارتباطات فعال و سازنده باشند و نه به عنوان مستندات سنگین و دائمی.
