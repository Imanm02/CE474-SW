\section*{جواب سوال ۷}

\section*{مقایسه وظایف مهارت مرور در سطح دوم و سوم}

\subsection*{سطح سوم \lr{(Entry Level)}}
در سطح سوم، فرد در حال یادگیری و اکتشاف ابتدایی فعالیت‌های مرتبط با مرور نرم‌افزار است. وظایف در این سطح عبارتند از:

\begin{enumerate}
	\item کمک به لجستیک‌های مورد نیاز برای بررسی‌ها و بازرسی‌ها، شامل:
	\begin{itemize}
		\item مدیریت لجستیک جلسه
		\item انجام داده‌کاوی مناسب
		\item تولید گزارش‌های مرتبط با جلسه
	\end{itemize}
	\item استفاده از چک‌لیست‌های مناسب که توسط برگزارکننده بررسی تعیین شده‌اند.
	\item جمع‌آوری داده‌های مناسب و دقیق که توسط برگزارکننده بررسی درخواست شده است.
\end{enumerate}

\subsection*{سطح دوم \lr{(Practitioner)}}
در سطح دوم، فرد مسئولیت‌های بیشتری در فرآیند مرور نرم‌افزار دارد و به طور فعال در بهبود کیفیت محصول مشارکت می‌کند. وظایف در این سطح شامل:

\begin{enumerate}
	\item شرکت فعال به عنوان عضو تیم بررسی برای دستیابی به اهداف فعالیت.
	\item شناسایی فرایندهای بررسی مناسب برای دستیابی به اهداف کیفیت محصول.
	\item شناسایی افراد مناسب برای مشارکت در فعالیت‌های بررسی.
	\item شناسایی اقدامات اصلاحی بر اساس داده‌های بررسی برای بهبود محصول در سراسر پروژه‌ها.
	\item تجزیه و تحلیل داده‌های محصول جمع‌آوری شده برای تجزیه و تحلیل علت ریشه‌ای و ارزیابی اثربخشی بررسی.
\end{enumerate}

تفاوت اصلی بین سطح سوم
\lr{(Entry Level)}
و سطح دوم
\lr{(Practitioner)}
در مهارت مرور نرم‌افزار، در نوع و میزان مسئولیت‌ها و نحوه مشارکت فرد در فرایندهای بررسی و بازبینی نرم‌افزار نهفته است. در سطح سوم، افراد عمدتاً در حال کسب تجربه و یادگیری اصول اولیه مرور نرم‌افزار هستند و به اجرای فرآیندها کمک می‌کنند. این سطح اغلب شامل فعالیت‌های پشتیبانی مانند آماده‌سازی مستندات، مدیریت لجستیک جلسات بررسی و جمع‌آوری داده‌های مرتبط با فرآیند است. نقش افراد در این سطح بیشتر تحت نظارت و راهنمایی افراد با تجربه‌تر انجام می‌گیرد تا با استانداردها، ابزارها و روش‌های مرور آشنا شوند.

در مقابل، در سطح دوم 
\lr{(Practitioner)}
افراد انتظار می‌رود که نقشی فعال‌تر و مستقیم‌تر در فرآیندهای بررسی داشته باشند. این شامل شرکت فعال در جلسات بررسی، شناسایی فرآیندهای بررسی مناسب، انتخاب افراد مناسب برای مشارکت در فعالیت‌های بررسی، و شناسایی و پیاده‌سازی اقدامات اصلاحی بر اساس نتایج بررسی‌ها است. در این سطح، افراد به طور فعال در تجزیه و تحلیل داده‌های مرتبط با کیفیت نرم‌افزار و ارزیابی اثربخشی فرآیندهای بررسی مشارکت می‌کنند. نقش آنها از اجرای ساده فرآیندها به سمت شناسایی فرصت‌های بهبود و اجرای تغییرات به منظور افزایش کیفیت نرم‌افزار و اثربخشی بررسی‌ها تغییر می‌کند. این سطح از مشارکت نه تنها به بهبود مستمر فرآیندها کمک می‌کند بلکه مهارت‌ها و دانش فردی افراد را در زمینه مرور و بازبینی نرم‌افزار نیز افزایش می‌دهد.

\section*{راهنمایی برای ارتقا به سطح \lr{Technical Leader} در مهارت مدیریت کیفیت نرم‌افزار}

برای ارتقا به سطح چهارم، آقای رسولی باید توجه ویژه‌ای به توسعه و پیاده‌سازی استراتژی‌های پیشرفته در مدیریت کیفیت نرم‌افزار داشته باشد. اقدامات کلیدی عبارتند از:

\begin{enumerate}
	\item \textbf{برقراری فرهنگ کیفیت:} ایجاد یک فرهنگ سازمانی که تولید محصولات با کیفیت و پیروی از فرایندهای کیفیت را در کانون توجه قرار دهد. این شامل برگزاری کارگاه‌های آموزشی، جلسات توجیهی و فراهم آوردن انگیزه‌هایی برای تیم‌ها برای رعایت استانداردهای کیفیت است.
	\item \textbf{تعیین استانداردها و فرایندهای کیفیت:} تعریف و اجرای استانداردها، مدل‌ها و فرایندهای مدیریت کیفیت به منظور اطمینان از دستیابی به اهداف کیفیتی. این شامل شناسایی و انتخاب مدل‌های کیفیت مناسب برای پروژه‌های مختلف و اطمینان از اجرای دقیق آنها توسط تیم‌ها است.
	\item \textbf{ایجاد فرایندهای جدید:} طراحی و پیاده‌سازی فرایندهای نوآورانه برای بهبود مستمر کیفیت محصولات و فرایندها. این ممکن است شامل توسعه ابزارهای جدید برای ارزیابی و تضمین کیفیت، یا روش‌های جدید برای اجرای بازرسی‌ها و تست‌ها باشد.
	\item \textbf{ارزیابی و بهبود مستمر:} انجام تجزیه و تحلیل مستمر برای ارزیابی اثربخشی فرایندهای مدیریت کیفیت موجود و شناسایی فرصت‌های بهبود. این شامل بررسی بازخوردهای دریافتی از تیم‌ها، مشتریان و سایر ذینفعان است.
	\item \textbf{طراحی ابزارها و فرایندها:} توسعه و پیشنهاد ابزارها و فرایندهای جدید که به افزایش به اهداف کیفیت محصول کمک می‌کنند. این شامل طراحی راهکارهای نرم‌افزاری خودکار برای تسهیل فرآیندهای تست و بازرسی، و همچنین ابزارهای مدیریتی برای پیگیری و گزارش‌دهی پیشرفت کیفیت می‌شود.
\end{enumerate}

این اقدامات به آقای رسولی کمک خواهد کرد تا در نقش خود به عنوان یک رهبر فنی در زمینه مدیریت کیفیت نرم‌افزار پیشرفت کند و به سطح بالاتری از شایستگی دست یابد. تمرکز بر رویکردهای نوآورانه و ارزیابی مستمر فرایندها، ابزارها و روش‌ها، نه تنها به بهبود مستمر کیفیت کمک می‌کند، بلکه به توسعه‌دهندگان و تیم‌های مهندسی امکان می‌دهد تا با اعتماد به نفس بیشتری بر روی پروژه‌های خود کار کنند. نقش آقای رسولی به عنوان یک Technical Leader نیازمند درک عمیقی از اصول مدیریت کیفیت، همراه با توانایی رهبری و الهام بخشی به دیگران برای پیروی از این اصول در تمامی جنبه‌های توسعه نرم‌افزار است.