\section*{جواب سوال ۴}

الگوی
\lr{Pipes and Filters} :

هدف:  
این الگو به تقسیم وظایف به مدولار مستقل کمک می کند تا خطوط ارتباطی (pipes) بین آنها ایجاد کند و داده ها را از یک فیلتر به دیگری منتقل کند. این امر باعث میشود که تغییر و تعمیر هر فیلتر بدون تاثیر بر دیگران انجام شود.

مزایا:

- خوانایی و تعمیرپذیری بهتر نرم افزار

- امکان تغییر و تعمیر هر فیلتر به طور مستقل

- امکان اعمال تغییرات بدون مرور کل سیستم


معایب:
 

- افزایش پیچیدگی برنامه به دلیل تعداد زیاد مرزها

- نیاز به مدیریت ترافیک داده ها بین فیلترها


حوزه کاربرد پیشنهادی:

- صنعت نرم افزار، پردازش تصاویر و فیلم ها

- پردازش متون و مستندات

- پردازش داده های حساس مانند مالی و پزشکی


با توجه به الگوی Pipes and Filters می‌توان گفت:

در این الگو سیستم به قطعات کوچک موسوم به فیلترها تقسیم می‌شود که هر فیلتر وظیفه‌ای خاص را انجام می‌دهد.

فیلترها به وسیله خطوط ارتباطی یا pipes به هم متصل می‌شوند تا داده‌ها را از یک فیلتر به فیلتر دیگر منتقل کنند.

فیلترها می‌توانند عملیاتی مانند فیلتر کردن، ترکیب، تجزیه و تحلیل و ... انجام دهند.

سیستم به شکل خطی قرار می‌گیرد و داده ها از یک سو شروع می‌شود و به ترتیب از فیلتری به فیلتر دیگر می‌گذرد.

هر فیلتر مستقل است و تغییر در آن تاثیری بر دیگر فیلترها ندارد.

امکان تغییر و تعمیر هر قطعه به صورت مستقل وجود دارد.

بنابراین این الگو ساختار مدولار و جدا از هم دارد که باعث انعطاف‌پذیری و تعمیرپذیری ساده‌تر می‌شود.

\newpage

الگوی
\lr{Broker} :

هدف:
ارتباط مستقیم بین ماژولها را از طریق واسطه ای به نام بروکر کنترل و مدیریت میکند تا کارایی بالاتر و قابلیت استفاده مجدد بیشتری داشته باشند.

مزایا:

- استقلال ماژولها و جلوگیری از وابستگی متقابل

- امکان تغییر و تعمیر ماژولها بدون تاثیر بر دیگران

- انعطاف پذیری و قابلیت استفاده مجدد بیشتر

معایب:

- افزایش پیچیدگی و تاخیر در ارتباطات به دلیل وجود واسطه

- نیاز به مدیریت واسطه

حوزه کاربرد پیشنهادی:

- سیستم های توزیع شده و موازی

- سیستم هایی با ارتباطات پیچیده

- سیستم های اشتراک منابع



الگوی Broker به شرح زیر است:

در این الگو ارتباط مستقیم میان ماژول‌های مختلف جلوگیری می‌شود.

به جای آن ارتباطات از طریق یک المان مرکزی به نام Broker برقرار می‌گردد.

بروکر نقش واسطه و کنترل‌کننده تردد پیام‌ها را بین ماژول‌ها برعهده دارد.

ماژول‌ها دیگر مستقیماً با هم ارتباط ندارند بلکه پیام‌ها را عبوری از بروکر می‌فرستند.

این کار باعث کاهش وابستگی‌ها و افزایش قابلیت تغییرپذیری می‌شود.

بروکر همچنین مسئولیت‌هایی مانند اعتبارسنجی، تعیین ترتیب پیام‌ها و تنظیم منابع را برعهده دارد.

در عوض افزایش پیچیدگی و تاخیر در ارتباطات ناشی از واسطه رخ می‌دهد.

بنابراین Broker الگویی مرکزی و واسطه‌ای برای ارتباط ماژول‌ها فراهم می‌کند.

\newpage

الگوی
\lr{Microkernel} :

هدف:  
جداسازی سرویس های اصلی سیستم عامل از هسته بسیار کوچک با هدف افزایش امنیت و قابلیت اطمینان.

مزایا:

- امنیت و پایداری بهتر سیستم عامل

- قابلیت تغییرپذیری و تعمیرپذیری آسان‌تر

- جلوگیری از خطاهای هسته

معایب:
- کاهش عملکرد به دلیل افزایش بارگذاری بین هسته و سرویس‌ها

- پیچیدگی بیشتر مدیریت منابع

حوزه کاربرد پیشنهادی:

- سیستم عامل های توکار و صنعتی با نیاز به امنیت بالا

- سیستم عامل های سرور بزرگ

- سیستم عامل های توزیع شده



الگوی Microkernel شامل موارد زیر است:

در این الگو هسته سیستم عامل بسیار ساده می‌شود و تنها وظایف بسیار اساسی مانند مدیریت پردازنده، حافظه، رابط با سخت‌افزار را برعهده دارد.

سایر سرویس‌های سیستم عامل مانند سیستم پرونده‌ها، شبکه، چاپگر و .. در قالب سرویس‌های کاربردی به هسته خارج می‌شوند.

ارتباط بین هسته و سرویس‌های کاربردی به صورت پیام‌رسانی 
\lr{(Message Passing)}
برقرار می‌شود که باعث افزایش امنیت و جداسازی می‌گردد.

سرویس‌های کاربردی به صورت اشیاء مستقل در می‌آیند که می‌توان آن‌ها را بدون تأثیر بر سایر مؤلفه‌ها تغییر داد.

این روش سبب افزایش قابلیت اعتماد، امنیت و تعمیرپذیری می‌شود.

هرچند به دلیل افزایش پیام، عملکرد سیستم کاهش می‌یابد.

بنابراین Microkernel ساختار باز و مدولار داشته که قابلیت تغییر و تعمیر را تسهیل کرده و امنیت را افزایش می‌دهد.