\section*{جواب سوال ۶}

\section{مقدمه}
متدولوژی
\lr{Extreme Programming (XP)}
با تاکید بر بهبود فرآیند توسعه نرم‌افزار و ارتقاء کیفیت محصول نهایی، استراتژی‌های مختلفی را برای مرور و بازبینی کد پیشنهاد می‌دهد. این متدولوژی چابک، که بر تعاملات تیمی، رضایت مشتری و تحویل مداوم محصولات با کیفیت تاکید دارد، فعالیت‌های مروری را به عنوان بخش مهمی از فرآیند توسعه در نظر می‌گیرد.

\section{فعالیت‌های مروری در \lr{XP}}
فعالیت‌های مروری در \lr{XP} شامل کدنویسی جفتی، توسعه محور تست (TDD) ، و بازبینی کد توسط همتیمی‌ها است. هر یک از این فعالیت‌ها به شیوه‌ای منحصر به فرد به ارتقاء کیفیت کمک می‌کند.

\subsection{کدنویسی جفتی}
کدنویسی جفتی، که در آن دو برنامه‌نویس به طور همزمان بر روی یک مسئله کار می‌کنند، از ایجاد اشتباهات جلوگیری کرده و به بهبود کیفیت کد کمک می‌کند. این فرآیند مروری، ارتباط و همفکری را در بین اعضای تیم تقویت می‌کند و به تسریع فرآیند شناسایی و رفع خطاها منجر می‌شود.

\subsection{توسعه محور تست (TDD)}
TDD
یک رویکرد سیستماتیک است که در آن توسعه‌دهندگان ابتدا تست‌هایی برای ویژگی‌های نرم‌افزار می‌نویسند و سپس کدی را پیاده‌سازی می‌کنند که این تست‌ها را پاس کند. این روش موجب می‌شود که توسعه‌دهندگان تمرکز بیشتری بر روی نیازمندی‌ها و کیفیت کد داشته باشند.

\subsection{بازبینی کد توسط همتیمی‌ها}
بازبینی‌های کد توسط همتیمی‌ها فرصتی برای ارزیابی و بهبود کد از دیدگاه‌های مختلف فراهم می‌آورد. این فرآیند به اشتراک‌گذاری دانش و بهترین شیوه‌ها کمک کرده و اطمینان حاصل می‌کند که کد نوشته شده با استانداردهای تیم سازگار است.

\section{اثرگذاری فعالیت‌های مروری بر کیفیت}
این فعالیت‌های مروری به طور مستقیم بر افزایش کیفیت نرم‌افزار تاثیر می‌گذارند. کدنویسی جفتی و TDD با کاهش خطاها و افزایش پوشش تست، اطمینان از ایجاد کد با کیفیت بالا را فراهم می‌آورند. بازبینی کد توسط همتیمی‌ها به حفظ یکپارچگی کد و اطمینان از رعایت استانداردهای تیم کمک می‌کند.

\section{چالش‌ها}
علی‌رغم مزایای بی‌شمار، پیاده‌سازی این فعالیت‌های مروری ممکن است با چالش‌هایی همراه باشد، از جمله مقاومت در برابر تغییر، نیاز به زمان بیشتر برای اجرای فعالیت‌ها، و نیاز به آموزش برای افزایش مهارت‌های تیم.

\section{نتیجه‌گیری}
فعالیت‌های مروری در متدولوژی \lr{XP} یک عنصر حیاتی برای تضمین کیفیت نرم‌افزار هستند. با اجرای موثر این استراتژی‌ها، تیم‌های توسعه می‌توانند محصولاتی با کیفیت بالاتر و با اطمینان بیشتری تولید کنند. در نهایت، تعهد به این فعالیت‌های مروری و حل چالش‌های مربوطه، می‌تواند به ساخت نرم‌افزارهایی دوام‌پذیر، قابل نگهداری و کارآمد منجر شود.