\section*{سوال ۳}

تفاوت فعالیت های تحلیل و طراحی سیستم های نرم‌افزاری را توضیح دهید. اطمینان حاصل کنید که در توضیحات خود به موارد زیر بپردازید:

\begin{itemize}
	\item ارتباط آن دو با یک مساله و راه حل آن
	\item اهداف و تمرکز هر یک
	\item سطح انتزاع هر کدام
	\item تقدم و تاخر هر یک از این دو فعالیت
	\item تفاوت مدل‌سازی ذیل هر فعالیت
\end{itemize}

\section*{جواب سوال ۳}

\section*{ارتباط با مسئله و راه‌حل}
\textbf{تحلیل:} در تحلیل نرم‌افزار، مسئله مورد بررسی قرار می‌گیرد. هدف این است که دقیقاً تعریف کنیم مسئله چیست و چه نیازهایی باید توسط نرم‌افزار برآورده شود. \\
\textbf{طراحی:} در مرحله طراحی، راه‌حل‌های ممکن برای مسائل تحلیل شده مطرح می‌شوند. این مرحله شامل تعیین چگونگی عملکرد نرم‌افزار برای برآورده کردن نیازهای شناسایی‌شده است.

\section*{اهداف و تمرکز}
\textbf{تحلیل:} تمرکز در تحلیل بر روی شناسایی و فهم نیازمندی‌های کاربر و مشخص کردن آنچه سیستم باید انجام دهد، است. \\
\textbf{طراحی:} هدف از طراحی ایجاد یک معماری قابل اجرا برای نرم‌افزار است که نیازمندی‌های تحلیل شده را پوشش دهد.

\section*{سطح انتزاع}
\textbf{تحلیل:} در تحلیل، سطح انتزاع بالاتر است. این مرحله بیشتر بر روی "چه" تمرکز دارد تا "چگونه". \\
\textbf{طراحی:} طراحی در سطح انتزاع پایین‌تر قرار دارد و بیشتر به جزئیات "چگونه" می‌پردازد.

\section*{تقدم و تاخر}
\textbf{تحلیل:} معمولاً قبل از طراحی انجام می‌شود. ابتدا باید مسائل و نیازمندی‌ها را درک کرد. \\
\textbf{طراحی:} پس از تحلیل انجام می‌شود و بر اساس نتایج به دست آمده از تحلیل، راه‌حل‌ها طراحی می‌شوند.

\section*{تفاوت در مدل‌سازی}
\textbf{تحلیل:} مدل‌سازی در تحلیل بر روی نمایش نیازمندی‌ها و فرایندهای کسب‌وکار تمرکز دارد. \\
\textbf{طراحی:} در طراحی، مدل‌سازی به توصیف معماری سیستم، کلاس‌ها، اشیاء، و روابط بین آن‌ها می‌پردازد.