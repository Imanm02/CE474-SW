\section*{جواب سوال ۵}

\title{تحلیل ویژگی‌های کیفی پوشش داده شده توسط تکنیک‌های معماری نرم‌افزار}
\author{}
\date{}

\section{مقدمه}
تکنیک‌های معماری نرم‌افزار نقش حیاتی در تعیین ویژگی‌های کیفی سیستم‌های نرم‌افزاری دارند. در این مقاله، به بررسی سه تکنیک معماری اصلی:
\lr{Discovery Service}
و
\lr{Redundant Spare}
و
\lr{Timestamp}
پرداخته و تاثیر آن‌ها بر ویژگی‌های کیفی مورد بررسی قرار می‌گیرد.

\section{\lr{Discovery Service}}
\lr{Discovery Service}
یا خدمت کشف، به سیستم‌های نرم‌افزاری امکان می‌دهد تا به طور خودکار سرویس‌ها و منابع موجود را در شبکه کشف کنند. این تکنیک به ویژه در معماری‌های میکروسرویسی که در محیط‌های ابری اجرا می‌شوند، اهمیت پیدا می‌کند.

\subsection*{کشف سرویس و تاثیر آن بر ویژگی‌های کیفی}
\begin{itemize}
	\item \textbf{قابلیت اطمینان:} 
	\lr{Discovery Service} با ارائه مکانیزمی برای کشف خودکار سرویس‌ها و منابع، اطمینان می‌دهد که سیستم‌ها و برنامه‌های کاربردی همواره قادر به یافتن و استفاده از منابع مورد نیاز خود هستند، حتی در صورت تغییر موقعیت یا دسترسی سرویس‌ها. این امر باعث می‌شود که سیستم نرم‌افزاری در مقابل اختلالات و تغییرات زیرساختی انعطاف‌پذیر و مقاوم باشد.
	
	\item \textbf{انعطاف‌پذیری:}
	\lr{Discovery Service} به سیستم‌ها اجازه می‌دهد تا به سرعت به تغییرات پاسخ دهند و منابع جدید را بدون نیاز به تنظیمات دستی کشف و مدیریت کنند. این تکنیک به ویژه برای سیستم‌های مبتنی بر میکروسرویس‌ها و زیرساخت‌های ابری که مدام در حال تحول هستند، ایده‌آل است و به آن‌ها امکان می‌دهد تا با کمترین وقفه به رشد و توسعه پایدار ادامه دهند..
\end{itemize}

\section{Redundant Spare}
Redundant Spare
به سه شکل عمده اجرا می‌شود:

\lr{active redundancy (hot spare), passive redundancy (warm spare)}, \lr{spare (cold spare)}.
این تکنیک به سیستم اجازه می‌دهد تا در صورت بروز خطا در یکی از اجزاء، به سرعت به جزء یدکی سوئیچ کند و به این ترتیب قابلیت اطمینان و دسترس‌پذیری سیستم را افزایش دهد.

\subsection*{Redundant Spare و تاثیر آن بر ویژگی‌های کیفی}
\begin{itemize} 
	\item \textbf{قابلیت اطمینان و دسترس‌پذیری:} 
	با استفاده از \lr{Redundant Spare} ، سیستم‌ها می‌توانند در صورت بروز خطا در یک جزء، به طور خودکار به یک جزء یدکی سوئیچ کنند، بدون اینکه کاربران اختلال قابل توجهی را تجربه کنند. این تکنیک با افزایش دسترس‌پذیری و کاهش زمان توقف، به ویژه برای سیستم‌های بحرانی که نیاز به دسترسی مداوم دارند، حیاتی است.
	
	
	\item \textbf{انعطاف‌پذیری در برابر خطاها:} 
	
	\lr{Redundant Spare} با فراهم آوردن سطوح مختلفی از پشتیبان‌گیری (فعال، غیرفعال و سرد) به سیستم‌ها امکان می‌دهد تا با انواع مختلف خطاها به شکل موثرتری مقابله کنند. این تکنیک به سیستم اجازه می‌دهد تا حتی در شرایط نامطلوب نیز به فعالیت خود ادامه دهد و از اختلالات گسترده جلوگیری کند.
\end{itemize}

\section{\lr{Timestamp}}
تکنیک
\lr{Timestamp}
 برای تشخیص ترتیب غلط رخدادها، به ویژه در سیستم‌های توزیع‌شده که از پیام‌رسانی استفاده می‌کنند، به کار می‌رود. این تکنیک با اختصاص یک مهر زمانی به هر رویداد، پس از وقوع آن، به تعیین ترتیب صحیح رخدادها کمک می‌کند.

\subsection*{\lr{Timestamp} و تاثیر آن بر ویژگی‌های کیفی}
\begin{itemize}
	\item \textbf{دقت و یکپارچگی داده‌ها:} استفاده از 
	\lr{Timestamp}
به همزمان‌سازی دقیق عملیات‌ها در سیستم‌های توزیع‌شده کمک می‌کند، به طوری که هر عملیات بر اساس زمان وقوع خود به ترتیب اجرا می‌شود. این کنترل دقیق بر ترتیب رویدادها از تداخل داده‌ها و ناسازگاری‌ها جلوگیری می‌کند و به حفظ دقت و هماهنگی در سیستم‌های پیچیده کمک می‌کند.

	\item \textbf{همزمان‌سازی:} تکنیک 
	\lr{Timestamp}
	در همزمان‌سازی عملیات‌ها در سیستم‌های توزیع‌شده نقش مهمی دارد و از اجرای نادرست عملیات‌ها بر اساس ترتیب زمانی غلط جلوگیری می‌کند.
\end{itemize}

\section{نتیجه‌گیری}
تکنیک‌های معماری نرم‌افزار مانند
\lr{Discovery Service}
و
\lr{Redundant Spare}
و
\lr{Timestamp}
نقش مهمی در بهبود ویژگی‌های کیفی سیستم‌های نرم‌افزاری دارند. هر یک از این تکنیک‌ها به شیوه‌ای منحصر به فرد به افزایش دسترس‌پذیری، قابلیت اطمینان، دقت داده‌ها، و همزمان‌سازی کمک می‌کنند. انتخاب درست این تکنیک‌ها بر اساس نیازهای خاص هر پروژه، می‌تواند تاثیر به سزایی در موفقیت نهایی سیستم داشته باشد.

منابع:

\begin{itemize}
	\item \lr{SEIBlog. (2021). Tactics and Patterns for Software Robustness.}
	\item \lr{Oroumand. (2021). Intro to Service Discovery.}
\end{itemize}

\bibliographystyle{apalike}
\bibliography{references}