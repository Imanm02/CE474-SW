\section*{سوال ۴}

چرا نیازمندی‌های پروژه این‌قدر تغییر می‌کنند؟ آیا مشتری نمی‌داند چه می‌خواهد؟!

\section*{جواب سوال ۴}

تغییرات در نیازمندی‌های پروژه معمولاً به دلیل چندین عامل مختلف رخ می‌دهند.


یکی از این عوامل می‌تواند عدم دقت در تعریف نیازمندی‌ها باشد. در ابتدای یک پروژه، مشتری ممکن است نتواند نیازمندی‌های خود را به طور کامل و دقیق تعریف کند. این ممکن است به دلیل عدم آگاهی کامل از قابلیت‌ها و محدودیت‌های سیستم باشد یا به دلیل عدم تجربه در حوزه فناوری اطلاعات. بنابراین، در طول زمان و با پیشرفت پروژه، مشتری ممکن است به نحوه بهتری بتواند نیازمندی‌های خود را تعریف کند و تغییرات لازم را اعمال کند.


عوامل دیگری مانند تغییر در شرایط کسب و کار، رقابت با سایر سازمان‌ها، تغییرات در فناوری و نیازهای جدید مشتریان نیز می‌توانند باعث تغییر در نیازمندی‌های پروژه شوند. در واقع، در دنیای امروز، بازار و تکنولوژی به سرعت تغییر می‌کنند. آنچه امروز نیاز است، ممکن است چند ماه دیگر منسوخ شود. شرکت‌ها برای بقا در بازار رقابتی باید خود را با این تغییرات هماهنگ کنند، که این امر می‌تواند به تغییر نیازمندی‌ها منجر شود.


همچنین، در برخی موارد، مشتری ممکن است در ابتدا نتواند تمامی جزئیات پروژه را پیش‌بینی کند و تغییرات لازم را در طول زمان اعمال کند. در واقع، فیدبک کاربران نهایی پس از مشاهده‌ی نسخه‌های اولیه‌ی محصول هم می‌تواند منجر به تغییرات در نیازمندی‌ها شود.


همچنین ممکن است محدودیت‌هایی در زمینه‌ی تکنولوژی یا معماری سیستم وجود داشته باشد که تنها در حین توسعه‌ی محصول مشخص شوند و نیازمند تغییر در نیازمندی‌ها باشند.


بعضا حتی تغییر در قوانین دولتی یا صنعتی نیز می‌تواند منجر به تغییر نیازمندی‌ها شود.


به طور کلی، تغییرات در نیازمندی‌های پروژه نشان از یک فرآیند تکاملی و تعاملی است که در طول زمان بهبود می‌یابد. این تغییرات معمولاً نشان از این دارند که مشتری بهتر متوجه نیازهای خود می‌شود و سعی می‌کند تا بهترین نتیجه را از پروژه بگیرد.
