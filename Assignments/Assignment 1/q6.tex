\section*{سوال ۶}

تفاوت «مدل ایجاد نرم‌افزار» مانند آبشاری یا حلزونی با «متدولوژی ایجاد نرم‌افزار» مانند XP یا RUP در چیست؟

انجمن علمی دانشکده مهندسی کامپیوتر خواستار «مدلی» برای برگزاری رویدادهای دانشجویی است. در طراحی این مدل، باید به ویژگی‌های زیر توجه شود:
\begin{itemize}
	\item حق‌الزحمه‌ای به نیروهای برگزارکننده پرداخت نمی‌شود.
	\item احتمال عدم انجام وظایف توسط برگزارکنندگان به دلیل عدم تعهد رسمی.
	\item دانشجویان وقت محدودی دارند.
	\item موضوعات رویداد حول مباحث رشته‌ی مهندسی کامپیوتر است.
	\item هدف اصلی، یادگیری و سپس لذت بردن از کار تیمی است.
	\item مخاطبین عمدتاً دانشجویان و دانش‌آموزان هستند.
\end{itemize}

\subsection*{موارد مورد توجه در طراحی}
\begin{itemize}
	\item جامعه مخاطبین
	\item ثبت‌نام مخاطبین
	\item جذب داوطلبین برگزاری
	\item انتخاب افراد داوطلب
	\item تخمین هزینه‌ها
	\item حامی مالی
	\item تبلیغات و برندینگ
	\item خط زمانی رویداد
	\item هماهنگی‌های اداری
\end{itemize}

با توجه به مدلی که در قسمت قبل تهیه کرده‌اید، متدولوژی‌ای برای برگزاری یک رویداد خاص طراحی کنید. این متدولوژی باید موقعیت خاصی را در نظر بگیرد و به صورت دقیق به ویژگی‌های آن بپردازد.

\textbf{توضیحات متدولوژی:} 
\begin{enumerate}
	\item تعریف موقعیت و ویژگی‌های رویداد
	\item تحلیل نیازمندی‌ها و محدودیت‌های رویداد
	\item برنامه‌ریزی جامع برای برگزاری
	\item پیاده‌سازی و اجرای رویداد
	\item ارزیابی و بازبینی پس از اتمام رویداد
\end{enumerate}

\section*{جواب سوال ۶}

تفاوت بین
\textbf{مدل ایجاد نرم‌افزار}
و
\textbf{متدولوژی ایجاد نرم‌افزار}
به نحوه‌ی دستورالعمل‌ها، فرآیندها، تکنیک‌ها و ابزارهایی برمی‌گردد که در هر کدام استفاده می‌شوند. بیایید این دو را با یکدیگر مقایسه کنیم:

\subsection*{مدل ایجاد نرم‌افزار:}

مدل ایجاد نرم‌افزار به الگوهای کلی مراحل و فعالیت‌های لازم برای توسعه نرم‌افزار اشاره دارد. این مدل‌ها معمولاً رویکردی سطح بالا به فرآیند توسعه نرم‌افزار دارند و می‌توانند مفاهیم مختلفی را در بر بگیرند که تیم‌ها باید دنبال کنند.

\begin{itemize}
	\item \textbf{آبشاری \lr{(Waterfall)} :}
یک مدل خطی و ترتیبی است که در آن هر مرحله باید کاملاً تمام شود قبل از اینکه مرحله بعدی شروع شود. مثال: ابتدا تحلیل نیازمندی‌ها، سپس طراحی سیستم، پس از آن پیاده‌سازی، تست و نهایتاً نگهداری.
	\item \textbf{حلزونی \lr{(Spiral)} :}
مدل حلزونی نیز مراحل آبشاری را دنبال می‌کند، اما با یک رویکرد تکراری که اجازه می‌دهد بازگشت به مراحل قبلی برای بهبود و اصلاح وجود داشته باشد. در هر دور، یک نسخه جدید و بهبود یافته از نرم‌افزار ساخته می‌شود.
\end{itemize}

\subsection*{متدولوژی ایجاد نرم‌افزار:}

متدولوژی ایجاد نرم‌افزار نه تنها مراحل کلی فرآیند توسعه را تعریف می‌کند، بلکه تکنیک‌ها، ابزارها، و دستورالعمل‌های دقیقی را برای هر مرحله ارائه می‌دهد. متدولوژی‌ها معمولاً بسیار جامع‌تر هستند و می‌توانند شامل توصیه‌هایی برای برنامه‌ریزی، تخمین زمان، مدیریت پروژه، توسعه و نگهداری باشند.

\begin{itemize}
	\item \textbf{\lr{XP (eXtreme Programming)} :}
یک متدولوژی چابک است که بر توسعه تکراری، برنامه‌ریزی مداوم، و بهبود مستمر تاکید دارد. همچنین، این متدولوژی بر توسعه به شیوه‌ی جفتی، تست محور و داشتن بازخورد مداوم از مشتری تأکید می‌کند.

	\item \textbf{\lr{RUP (Rational Unified Process)} :}
این متدولوژی یک فرآیند تکراری و افزایشی است که به تیم‌ها کمک می‌کند تا معماری نرم‌افزار را به خوبی تعریف کنند و مدیریت ریسک را در فرآیند توسعه ادغام کنند. RUP مجموعه‌ای از بهترین شیوه‌ها را در تمام جنبه‌های توسعه نرم‌افزار معرفی می‌کند.
\end{itemize}

بنابراین با توجه به تعریف‌هایی که داشتیم، تفاوت عمده در این است که مدل‌های توسعه نرم‌افزار بیشتر به الگوی کلی و توالی فعالیت‌ها توجه دارند، در حالی که متدولوژی‌های توسعه نرم‌افزار جزئیات دقیق‌تری از نحوه اجرای هر مرحله و اصول راهنما را ارائه می‌دهند و اغلب شامل راهنمایی‌های عملی‌تر و مشخص‌تر برای تیم‌های توسعه می‌شوند.

\begin{itemize}
	\item \textbf{تمرکز بر فرآیند:} مدل‌های ایجاد نرم‌افزار بیشتر روی فرآیند توسعه متمرکز هستند. آنها مراحل و توالی عمومی فعالیت‌های مورد نیاز برای تولید نرم‌افزار را تعریف می‌کنند.
	\item \textbf{جامعیت پایین‌تر:} مدل‌هایی مانند آبشاری یا حلزونی معمولاً دستورالعمل‌های مشخص و جزئی برای پیاده‌سازی فرآیندها ارائه نمی‌دهند. آن‌ها چارچوب‌های کلی هستند که نحوه به دنبال کردن هر مرحله را به تیم‌های توسعه واگذار می‌کنند.
	\item \textbf{انعطاف‌پذیری کمتر:} مدل‌ها مانند آبشاری سفت و سخت‌تر هستند و تغییرات را در میانه‌ی پروژه به خوبی تحمل نمی‌کنند.
	\item \textbf{پیش‌بینی‌پذیری:} این مدل‌ها به دلیل ترتیب مشخص شده‌شان پیش‌بینی‌پذیری بیشتری در مراحل توسعه فراهم می‌آورند، که می‌تواند برای مدیریت پروژه مفید باشد.
\end{itemize}

\subsection*{بررسی دقیق‌تر متدولوژی‌های ایجاد نرم‌افزار (مانند XP یا RUP ):}

\begin{itemize}
	\item \textbf{تمرکز بر جزئیات:} متدولوژی‌ها جزئیات دقیق‌تری از نحوه اجرای هر مرحله از فرآیند توسعه را فراهم می‌آورند، شامل روش‌ها، ابزارها، و دستورالعمل‌های خاص.
	\item \textbf{جامعیت بالاتر:} متدولوژی‌ها مجموعه‌ای از بهترین شیوه‌ها، قالب‌ها و استانداردهای صنعتی را ادغام می‌کنند که می‌تواند شامل توصیه‌های متعدد برای تمام جنبه‌های توسعه نرم‌افزار باشد.
	\item \textbf{انعطاف‌پذیری بیشتر:} متدولوژی‌ها مانند XP طراحی شده‌اند تا به تیم‌ها اجازه دهند به صورت چابک و با قابلیت پاسخگویی بالا به تغییرات پاسخ دهند.
	\item \textbf{تاکید بر بهبود مداوم:} متدولوژی‌ها اغلب شامل مکانیزم‌هایی برای بازنگری و بهبود مداوم فرآیندها هستند.
\end{itemize}

\subsubsection*{مثال:}

\begin{itemize}
	\item مدل آبشاری به شما می‌گوید که ابتدا نیازمندی‌ها را جمع‌آوری کنید، سپس طراحی کنید، پس از آن کدنویسی، سپس تست و در نهایت به تحویل محصول بپردازید. این یک توالی خطی و غیرقابل بازگشت است.
	\item متدولوژی RUP ، که یک متدولوژی تکراری و تدریجی است، به شما می‌گوید که چگونه باید نیازمندی‌ها را با استفاده از تکنیک‌های خاص جمع‌آوری کنید، چطور باید معماری را مدل‌سازی کنید، چگونه ریسک‌ها را مدیریت کنید و چطور فرآیندهای کدنویسی و تست را به صورت تکراری و با ادغام تغییرات انجام دهید.
\end{itemize}

\subsection*{ارائه‌ی مدل برای برگزاری رویدادهای انجمن علمی}

