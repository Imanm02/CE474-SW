\section*{سوال ۲}

تفاوت بین «مهندس نرم‌افزار» و «توسعه‌دهنده نرم‌افزار» چیست؟

\section*{جواب سوال ۲}

\section*{فعالیت‌ها}
توسعه‌دهنده نرم‌افزار بخش مشخصی از یک نرم‌افزار یا سیستم کامپیوتری را توسعه می‌دهد، ولی مهندس نرم‌افزار، برنامه‌ریزی، طراحی، توسعه و ساخت کامل یک نرم‌افزار یا سیستم کامپیوتری را بر عهده دارد و همچنین بر اساس اصول مهندسی و معیارهای کیفی، به مدیریت پروژه، تضمین کیفیت، تخمین هزینه و جدول‌بندی زمانی می‌پردازد.

\section*{مهارت‌ها}
\subsection*{مهارت‌های توسعه‌دهنده نرم‌افزار}
دانش زبان‌های برنامه‌نویسی فرانت‌اند و بک‌اند، تست کد، استفاده از ابزارهای نرم‌افزاری جهت توسعه برنامه‌های کاربردی، و درک خوبی از فرآیندهای توسعه چابک و یکپارچه‌سازی مداوم.

\subsection*{مهارت‌های مهندس نرم‌افزار}
دانش گسترده از زبان‌های برنامه‌نویسی فرانت‌اند و بک‌اند، دیباگ نرم‌افزار، ایجاد ابزارهای نرم‌افزاری، دانش معماری نرم‌افزار، مهارت همکاری، توانایی درک و تحلیل نیازمندی‌های کسب‌وکار و تبدیل آن‌ها به مشخصات فنی، و همچنین مهارت‌های مربوط به مدیریت پروژه و ارتباط با مشتری.

\section*{همکاری با دیگران}
توسعه‌دهنده معمولا به طور مستقل کار می‌کند و بعضا با توسعه‌دهنده‌های دیگر همکاری می‌کند. مهندس نرم‌افزار در یک محیط کاملا مشارکتی کار می‌کند و با سایر مهندسان، علاوه بر آن با تیم‌های دیگر مثل محصول، طراحی، تضمین کیفیت، و حتی مشتریان برای تضمین اینکه محصول نهایی نیازهای آن‌ها را برآورده می‌کند، مشارکت دارد.

\section*{نتیجه‌گیری}
به طور کلی توسعه‌دهنده روی یک بخش از سیستم یا برنامه تمرکز می‌کند در حالی که مهندس نرم‌افزار بر کل پروژه مسئولیت دارد. (می‌توانیم بگوییم انگار توسعه‌دهنده روی هر کدام از طبقات تمرکز دارد، ولی مهندس نرم‌افزار به کل ساختمان و ارتباط و تاثیر تمام طبقات به یکدیگر و در نهایت، کیفیت کل ساختمان توجه دارد، و به نوعی نقش معمار را در پروسه ساخت نرم‌افزار ایفا می‌کند.)