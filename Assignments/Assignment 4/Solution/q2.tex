\section*{جواب سوال ۲}

\section*{عبارات منطقی معادل برنامه‌ها}

\subsection*{1. انتخاب عدد میانی از بین سه عدد}
فرض کنید سه عدد \(a\) , \(b\) , و \(c\) داریم. عدد میانی \(m\) به صورت زیر تعریف می‌شود:
\[
m = \begin{cases} 
	a & \text{اگر } (a > b \land a < c) \lor (a < b \land a > c) \\
	b & \text{اگر } (b > a \land b < c) \lor (b < a \land b > c) \\
	c & \text{اگر } (c > a \land c < b) \lor (c < a \land c > b)
\end{cases}
\]

\subsection*{2. تشخیص زوج یا فرد بودن یک عدد}
یک عدد \(n\) زوج است اگر:
\[
\text{زوج}(n) \Leftrightarrow n \mod 2 = 0
\]
و فرد است اگر:
\[
\text{فرد}(n) \Leftrightarrow n \mod 2 \neq 0
\]

\subsection*{3. قدر مطلق تفاضل میان دو عدد}
قدر مطلق تفاضل دو عدد \(x\) و \(y\) به صورت زیر است:
\[
| x - y | = \begin{cases} 
	x - y & \text{اگر } x > y \\
	y - x & \text{اگر } y \geq x
\end{cases}
\]