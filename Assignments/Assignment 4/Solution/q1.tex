\section*{سوال ۱}

یک قانون سرانگشتی در فاز تحلیل این است که «افراد تیم ایجاد در فاز تحلیل باید بر نیازمندی‌هایی تمرکز کنند که در حوزه‌ی مسئله و کسب و کار قرار دارد».
\begin{enumerate}
	\item چه نوع نیازمندی‌هایی در این حوزه‌ها نیستند؟
	\item مثال بزنید.
\end{enumerate}

\section*{جواب سوال ۱}


\section*{مقدمه}
در فاز تحلیل مهندسی نرم‌افزار، تمرکز اصلی بر شناسایی و تعریف نیازمندی‌های کاربردی است که مستقیماً به حوزه‌ی مسئله و کسب و کار مرتبط هستند. با این حال، برخی نیازمندی‌ها وجود دارند که معمولاً در این فاز در نظر گرفته نمی‌شوند.

\section*{نیازمندی‌های غیرمرتبط}
\begin{enumerate}
	
	\item \textbf{نیازمندی‌های غیرعملکردی:} این نیازمندی‌ها شامل مواردی مانند امنیت، پایداری، کارایی و استانداردهای کیفی می‌شوند. به عنوان مثال، الزامات امنیتی یا زمان پاسخ سیستم. این نیازمندی‌ها بیشتر به چگونگی ارائه سرویس توسط سیستم مربوط می‌شود تا خود سرویس
	
	\item \textbf{نیازمندی‌های فنی:} این‌ها شامل انتخاب‌های فناورانه مانند پلتفرم‌های سخت‌افزاری و نرم‌افزاری، زبان‌های برنامه‌نویسی و ابزارهای توسعه می‌شوند. این نیازمندی‌ها بیشتر به راه‌حل فنی برای تحقق نیازمندی‌های کاربردی مربوط می‌شوند.
	
	\item \textbf{نیازمندی‌های مدیریتی یا سازمانی:} این نیازمندی‌ها به فرایندهای داخلی سازمانی، رویه‌های مدیریت پروژه و سیاست‌های کلان سازمانی مربوط می‌شوند. به عنوان مثال، نیازمندی‌هایی مانند رعایت استانداردهای خاص یا روش‌های گزارش‌دهی.
	
\end{enumerate}

\section*{مثال‌ها}
\begin{itemize}
	\item نیازمندی غیرعملکردی: در نظر گرفتن استانداردهای امنیتی بالا برای یک سیستم بانکی آنلاین که باید تراکنش‌ها را به شکل امن انجام دهد.
	
	\item نیازمندی فنی: استفاده از یک پایگاه داده خاص مانند MySQL به دلیل تجربه قبلی تیم توسعه در استفاده از این فناوری.
	
	\item نیازمندی مدیریتی: توسعه نرم‌افزار با استفاده از روش Agile به دلیل نیاز سازمان به انعطاف‌پذیری بالا و بازخورد سریع از کاربران.
	
\end{itemize}