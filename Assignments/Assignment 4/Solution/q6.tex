\section*{جواب سوال ۶}

جواب این سوال خیلی گسترده‌ست، سعی می‌کنم تا جای ممکن خلاصه جواب را بنویسم.

\subsection*{اهداف کیفیت نرم‌افزار}
پرسمن در فصل ۲۱ چهار هدف اصلی برای کیفیت نرم‌افزار پیشنهاد می‌دهد:
\begin{enumerate}
	\item \textbf{کارایی}: عملکرد نرم‌افزار در زمان اجرا و سرعت پاسخگویی به کاربر.
	\item \textbf{قابلیت اطمینان}: توانایی نرم‌افزار برای انجام دقیق وظایف مورد نظر بدون خطا و نقص.
	\item \textbf{قابلیت استفاده}: سهولت استفاده از نرم‌افزار و رابط کاربری آن.
	\item \textbf{قابلیت نگهداری}: توانایی نرم‌افزار برای به‌روزرسانی و نگهداری آسان.
\end{enumerate}

\subsection*{صفات و متریک‌های کیفی}
برای هر هدف، دو صفت و یک متریک کیفی تعیین می‌کنیم:

\textbf{کارایی}:
\begin{itemize}
	\item صفت: زمان پاسخ - متریک: میانگین زمان پاسخ به درخواست‌ها.
	\item صفت: استفاده از منابع - متریک: درصد استفاده از CPU و حافظه در حالت بار کامل.
\end{itemize}

\textbf{قابلیت اطمینان}:
\begin{itemize}
	\item صفت: نرخ خطا - متریک: تعداد خطاها به ازای هر هزار خط کد.
	\item صفت: زمان بازیابی پس از خطا - متریک: میانگین زمان لازم برای بازیابی سیستم پس از خطا.
\end{itemize}

\section*{قابلیت اطمینان}

\subsection*{میانگین زمان تا شکست (MTTF) و میانگین زمان تا بازیابی (MTTR)}
\begin{itemize}
	\item \textbf{MTTF}: میانگین زمانی که سیستم بدون خطا کار می‌کند.
	\item \textbf{MTTR}: میانگین زمان لازم برای تعمیر یا بازیابی سیستم پس از یک خطا.
\end{itemize}

\subsection*{میانگین زمان بین شکست‌ها (MTBF)}
\textbf{MTBF}: 
میانگین زمان بین شکست‌ها، که نشان‌دهنده دوره زمانی متوسط بین وقوع دو خطا در سیستم است. این شاخص برای ارزیابی قابلیت اطمینان و پایداری یک سیستم مورد استفاده قرار می‌گیرد. MTBF به عنوان یک معیار کلیدی در تضمین کیفیت نرم‌افزار، به سازمان‌ها کمک می‌کند تا درک بهتری از عملکرد محصولات خود در طول زمان داشته باشند و برنامه‌ریزی دقیق‌تری برای نگهداری و بهبود محصولات انجام دهند. محاسبه MTBF به صورت زیر انجام می‌شود:

\begin{equation}
	MTBF = \frac{\text{کل زمان کارکرد سیستم}}{\text{تعداد کل شکست‌ها}}
\end{equation}

این معیار به ویژه برای سیستم‌هایی که نیازمند دسترسی مداوم و قابلیت اطمینان بالا هستند، مانند سیستم‌های مالی، پزشکی، و حیاتی دیگر، اهمیت فراوانی دارد. افزایش MTBF نشان‌دهنده کاهش تعداد خطاها و افزایش قابلیت اطمینان سیستم است.

\section*{محاسبات مربوط به قابلیت اطمینان}

برای محاسبه میانگین زمان تا شکست (MTTF) ، میانگین زمان تا بازیابی (MTTR)، و میانگین زمان بین شکست‌ها (MTBF) ، ابتدا باید داده‌های مربوط به زمان‌های خرابی و بازیابی را جمع‌آوری کنیم. سپس با استفاده از فرمول‌های زیر این مقادیر را محاسبه می‌کنیم:

\begin{equation}
	MTTF = \frac{\text{کل زمان کارکرد بدون خطا}}{\text{تعداد کل شکست‌ها}}
\end{equation}

\begin{equation}
	MTTR = \frac{\text{کل زمان صرف شده برای بازیابی پس از شکست‌ها}}{\text{تعداد کل شکست‌ها}}
\end{equation}

\begin{equation}
	MTBF = MTTF + MTTR
\end{equation}

\subsection*{دسترسی‌پذیری سیستم}
دسترسی‌پذیری سیستم نشان‌دهنده درصد زمانی است که سیستم در دسترس و قابل استفاده است. این معیار را می‌توان با استفاده از فرمول زیر محاسبه کرد:

\begin{equation}
	\text{دسترسی پذیری} = \left( \frac{MTTF}{MTTF + MTTR} \right) \times 100
\end{equation}

\subsection*{معیار شکست بر زمان (FIT)}
معیار FIT نشان‌دهنده تعداد خطاهایی است که در هر یک میلیارد ساعت کارکرد سیستم رخ می‌دهد. این معیار به صورت زیر محاسبه می‌شود:

\begin{equation}
	FIT = \frac{10^9}{MTBF}
\end{equation}

\subsection*{داده‌ها:}
\begin{itemize}
	\item زمان کل خرابی‌ها = $1.5$ + $9.5$ + $14$ + $11$ = $36$ ساعت
	\item تعداد شکست‌ها = 4
	\item زمان کل کارکرد سیستم از 1 تا 30 آذر = 30 روز
\end{itemize}

\subsection*{محاسبات:}
\begin{itemize}
	\item \textbf{MTTF (میانگین زمان تا شکست):}
	\begin{itemize}
		\item کل زمان کارکرد بدون خطا = (30 روز $\times$ 24 ساعت) - 36 ساعت خرابی
		\item MTTF = $\frac{\text{کل زمان کارکرد بدون خطا}}{\text{تعداد شکست‌ها}}$
	\end{itemize}
	
	\item \textbf{MTTR (میانگین زمان تا بازیابی):}
	\begin{itemize}
		\item MTTR = $\frac{\text{کل زمان خرابی}}{\text{تعداد شکست‌ها}} = \frac{36 \text{ ساعت}}{4}$
	\end{itemize}
	
	\item \textbf{MTBF (میانگین زمان بین شکست‌ها):}
	\begin{itemize}
		\item MTBF = MTTF + MTTR
	\end{itemize}
	
	\item \textbf{دسترسی‌پذیری سیستم:}
	\begin{itemize}
		\item دسترسی‌پذیری = $\left( \frac{MTTF}{MTTF + MTTR} \right) \times 100$
	\end{itemize}
	
	\item \textbf{FIT (شکست بر زمان):}
	\begin{itemize}
		\item FIT = $\frac{10^9}{MTBF}$
	\end{itemize}
\end{itemize}

با فرض که زمان کل کارکرد سیستم 720 ساعت (30 روز $\times$ 24 ساعت) باشد و کل زمان خرابی 36 ساعت است، می‌توانیم محاسبات را انجام دهیم:

\begin{itemize}
	\item کل زمان کارکرد بدون خطا = $720 - 36 = 684$ ساعت
	\item MTTF = $\frac{684 \text{ ساعت}}{4} = 171 \text{ ساعت}$
	\item MTTR = $\frac{36 \text{ ساعت}}{4} = 9 \text{ ساعت}$
	\item MTBF = 171 + 9 = 180 ساعت
	\item دسترسی‌پذیری = $\left( \frac{171}{171 + 9} \right) \times 100 = تقریبا 95\%$
	\item FIT = $\frac{10^9}{180} = تقریبا 5,555,556$
\end{itemize}

این محاسبات به ما نشان می‌دهند که سیستم به طور متوسط هر 180 ساعت یک بار دچار شکست می‌شود، دسترسی‌پذیری آن تقریبا
 $95\%$
 است، و معیار FIT نشان می‌دهد که در هر یک میلیارد ساعت کارکرد، تقریبا 5.5 میلیون خطا رخ می‌دهد.

\section*{استانداردهای تضمین کیفیت}

استانداردهای تضمین کیفیت مانند \lr{ISO 9000} با هدف ایجاد چارچوبی برای سیستم‌های مدیریت کیفیت طراحی شده‌اند تا اطمینان حاصل شود که سازمان‌ها قادر به تولید محصولات و خدماتی هستند که به طور مداوم نیازهای مشتریان و الزامات قانونی را برآورده می‌سازند. دریافت استاندارد کیفیت \lr{ISO 9000} مستلزم این است که سازمان‌ها فرآیندهای خود را مطابق با الزامات استاندارد تنظیم و اجرا کنند و توانایی خود را در رعایت این استانداردها از طریق ارزیابی‌ها و بازرسی‌های دوره‌ای نشان دهند.

\section*{نقش تضمین کیفیت در تیم‌های توسعه}

نقش تضمین کیفیت (QA) در تیم‌های توسعه فراتر از انجام آزمون‌های دستی پیش از استقرار محصول است. تضمین کیفیت به عنوان یک فرآیند جامع عمل می‌کند که از برنامه‌ریزی، طراحی، توسعه، و تست محصول تا نگهداری آن پس از عرضه را شامل می‌شود. این فرآیند مستلزم همکاری نزدیک بین تیم‌های توسعه، تست، و عملیات است تا اطمینان حاصل شود که محصول نهایی نه تنها بدون خطا است بلکه به نیازهای کاربران نیز پاسخ می‌دهد. تضمین کیفیت به معنای واقعی کلمه در تمام مراحل چرخه زندگی نرم‌افزار دخیل است و شامل فعالیت‌هایی مانند تحلیل نیازمندی‌ها، طراحی آزمون‌ها، اجرای آزمون‌های خودکار و دستی، و بازنگری کد است.

اردهای تضمین کیفیت و نقش تضمین کیفیت در تیم‌های توسعه نرم‌افزار را ارائه می‌دهد. متن تکمیلی در این بخش به تشریح اهمیت استانداردهای بین‌المللی مانند \lr{ISO 9000} در ارتقاء کیفیت محصولات و خدمات نرم‌افزاری و همچنین به تاکید بر این موضوع می‌پردازد که چگونه یک فرآیند تضمین کیفیت جامع و همه‌جانبه می‌تواند به توسعه محصولات نرم‌افزاری با کیفیت بالا و رضایتمندی کاربران کمک کند.

در این بخش همچنین بر اهمیت همکاری میان تیم‌های مختلف در فرآیند توسعه نرم‌افزار تاکید شده است. تضمین کیفیت نباید تنها به عنوان یک فعالیت پایانی در نظر گرفته شود، بلکه باید به عنوان بخشی از تمام مراحل چرخه توسعه نرم‌افزار، از جمله تحلیل نیازمندی‌ها، طراحی، پیاده‌سازی و تست، ادغام شود. این رویکرد همه‌جانبه به تضمین کیفیت کمک می‌کند تا از بروز مشکلات در مراحل پایانی پروژه و هزینه‌های اضافی ناشی از آن‌ها جلوگیری کند.

همچنین، نقش افراد متخصص در تضمین کیفیت باید فراتر از انجام آزمون‌های دستی ساده باشد و شامل فعالیت‌هایی مانند تحلیل ریسک، مدیریت تغییر، بهبود فرآیندها و اطمینان از رعایت استانداردها و بهترین شیوه‌ها در تمام مراحل توسعه نرم‌افزار شود. این امر مستلزم دانش فنی عمیق، مهارت‌های ارتباطی قوی و درک کاملی از اهداف کسب‌وکار و نیازهای کاربران است.

از طریق اجرای دقیق استانداردهای تضمین کیفیت و تعهد به یک فرآیند تضمین کیفیت جامع، سازمان‌ها و تیم‌های توسعه می‌توانند به بهبود مستمر کیفیت محصولات نرم‌افزاری و افزایش رضایتمندی کاربران دست یابند.