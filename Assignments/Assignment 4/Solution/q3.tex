\section*{جواب سوال ۳}

\section*{سوال ۱: مقایسه مفاهیم fault، error، و failure}

\lr{Fault} (عیب)
،
\lr{Error} (خطا)
و
\lr{Failure}
(شکست)، سه مفهوم اصلی در تست نرم‌افزار هستند که برای درک عمیق‌تر علل و پیامدهای بروز مشکل در نرم‌افزارها، مورد بررسی قرار می‌گیرند.

\begin{itemize}
	\item \lr{Fault} (عیب) به مشکل یا نقص موجود در کد نرم‌افزار یا در مستندات آن اشاره دارد که می‌تواند زمینه‌ساز بروز خطا شود. این نقص‌ها به دلیل تصمیمات طراحی یا پیاده‌سازی نادرست توسط توسعه‌دهندگان نرم‌افزار ایجاد می‌شوند.
	
	\item \lr{Error} (خطا) وقتی اتفاق می‌افتد که نرم‌افزار در حین اجرا با عیب 
	\lr{(Fault)}
	مواجه شده و به دلیل آن، حالت داخلی نرم‌افزار از حالت مورد انتظار منحرف می‌شود. خطا لزوماً به معنای بروز شکست نیست اما نشان‌دهنده انحراف از رفتار مطلوب است.
	
	\item \textit{Failure} 
(شکست) زمانی رخ می‌دهد که نرم‌افزار نتواند عملکرد مورد انتظار را ارائه دهد و خروجی نرم‌افزار با خروجی مورد انتظار مطابقت نداشته باشد. شکست، نتیجه مشاهده شده از بروز یک یا چند خطا است.
\end{itemize}

\section*{سوال ۲: مدل \lr{RIPR}}

مدل \lr{RIPR} یک چارچوب برای فهم و تحلیل فرآیند بروز شکست در نرم‌افزار است و شامل چهار شرط اصلی است: \lr{Reachability} (قابلیت دسترسی)، \lr{Infection} (آلودگی)، \lr{Propagation} (انتشار)، و \lr{Revelation} (آشکارسازی).

\begin{enumerate}
	\item \lr{Reachability} (قابلیت دسترسی): تست باید قادر باشد نقطه‌ای از کد که عیب در آن وجود دارد را فعال ک
	\item \lr{Infection} (آلودگی): پس از رسیدن به نقطه‌ای که عیب در آن وجود دارد، حالت برنامه باید به گونه‌ای تغییر کند که نشان‌دهنده وجود خطا باشد.
	
	\item \lr{Propagation} (انتشار): حالت آلوده باید از طریق اجرای برنامه منتشر شود تا به خروجی یا وضعیت نهایی برنامه برسد و آن را به گونه‌ای اشتباه تحت تأثیر قرار دهد.
	
	\item \lr{Revelation} (آشکارسازی): تست باید قادر به آشکارسازی شکست باشد، به این معنا که شکست باید به روشنی قابل مشاهده و قابل تشخیص باشدند.

	\item \lr{Infection} (آلودگی): پس از رسیدن به نقطه‌ای که عیب در آن وجود دارد، حالت برنامه باید به گونه‌ای تغییر کند که نشان‌دهنده وجود خطا باشد.
	
	\item \lr{Propagation} (انتشار): حالت آلوده باید از طریق اجرای برنامه منتشر شود تا به خروجی یا وضعیت نهایی برنامه برسد و آن را به گونه‌ای اشتباه تحت تأثیر قرار دهد.
	
	\item \lr{Revelation} (آشکارسازی): تست باید قادر به آشکارسازی شکست باشد، به این معنا که شکست باید به روشنی قابل مشاهده و قابل تشخیص باشد.
\end{enumerate}

این مدل نشان می‌دهد که برای مشاهده یک شکست، همه این چهار شرط باید برآورده شوند. این یک فرآیند پیچیده است که نشان‌دهنده چالش‌های موجود در تست نرم‌افزار و اهمیت طراحی دقیق تست‌ها برای شناسایی عیب‌ها است.

\section*{سوال ۳: فرآیند \lr{Model-driven Test Design}}

\lr{Model-driven Test Design}
(طراحی تست مبتنی بر مدل) یک رویکرد سیستماتیک برای تولید موارد تست است که در آن مدل‌های فرمال از سیستم به عنوان اساس برای طراحی تست‌ها استفاده می‌شوند. این رویکرد به توسعه‌دهندگان اجازه می‌دهد تا پوشش دهی تست‌ها را به طور دقیق‌تری کنترل کنند و اطمینان حاصل کنند که تمام جنبه‌های مهم سیستم تست شده‌اند.

فرآیند طراحی تست مبتنی بر مدل عموماً شامل مراحل زیر است:

\begin{enumerate}
	
\item \textit{ایجاد مدل}: توسعه مدل‌هایی که جنبه‌های مختلف سیستم را نشان می‌دهند، مانند رفتار، ساختار، وابستگی‌ها، و سناریوهای استفاده.\item \textit{تعریف معیارهای پوشش}: تعیین معیارهایی برای ارزیابی کامل بودن تست‌ها، مانند پوشش دستورات، شرایط، یا مسیرهای اجرایی.

\item \textit{تولید موارد تست}: استفاده از مدل‌ها برای تولید خودکار موارد تست که بر اساس معیارهای پوشش تعریف شده هستند.

\item \textit{اجرای تست و تحلیل نتایج}: اجرای تست‌ها بر روی سیستم و تحلیل نتایج برای شناسایی و رفع عیب‌ها. این مرحله شامل بررسی نتایج تست‌ها برای تعیین اینکه آیا سیستم مطابق با مشخصات و الزامات عمل کرده است یا خیر.

\end{enumerate}

این رویکرد به تیم‌های توسعه امکان می‌دهد تا تست‌های دقیق و کامل‌تری را با استفاده از منابع محدود انجام دهند. با تمرکز بر مدل‌ها، تسترها می‌توانند اطمینان حاصل کنند که تمام جنبه‌های مهم سیستم تحت پوشش قرار گرفته‌اند و احتمال عبور عیب‌ها از فیلتر تست‌ها را به حداقل می‌رسانند.

استفاده از 
\lr{Model-driven Test Design}
نیازمند درک عمیقی از سیستم و توانایی تبدیل دانش به مدل‌های دقیق و قابل استفاده برای تست است. این رویکرد همچنین به ابزارهای تخصصی برای تولید خودکار موارد تست از مدل‌ها و اجرای آن‌ها بر روی سیستم نیاز دارد.