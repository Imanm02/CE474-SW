\section*{سوال ۶}

تفاوت «مدل ایجاد نرم‌افزار» مانند آبشاری یا حلزونی با «متدولوژی ایجاد نرم‌افزار» مانند XP یا RUP در چیست؟

انجمن علمی دانشکده مهندسی کامپیوتر خواستار «مدلی» برای برگزاری رویدادهای دانشجویی است. در طراحی این مدل، باید به ویژگی‌های زیر توجه شود:
\begin{itemize}
	\item حق‌الزحمه‌ای به نیروهای برگزارکننده پرداخت نمی‌شود.
	\item احتمال عدم انجام وظایف توسط برگزارکنندگان به دلیل عدم تعهد رسمی.
	\item دانشجویان وقت محدودی دارند.
	\item موضوعات رویداد حول مباحث رشته‌ی مهندسی کامپیوتر است.
	\item هدف اصلی، یادگیری و سپس لذت بردن از کار تیمی است.
	\item مخاطبین عمدتاً دانشجویان و دانش‌آموزان هستند.
\end{itemize}

\subsection*{موارد مورد توجه در طراحی}
\begin{itemize}
	\item جامعه مخاطبین
	\item ثبت‌نام مخاطبین
	\item جذب داوطلبین برگزاری
	\item انتخاب افراد داوطلب
	\item تخمین هزینه‌ها
	\item حامی مالی
	\item تبلیغات و برندینگ
	\item خط زمانی رویداد
	\item هماهنگی‌های اداری
\end{itemize}

با توجه به مدلی که در قسمت قبل تهیه کرده‌اید، متدولوژی‌ای برای برگزاری یک رویداد خاص طراحی کنید. این متدولوژی باید موقعیت خاصی را در نظر بگیرد و به صورت دقیق به ویژگی‌های آن بپردازد.

\section*{جواب سوال ۶}

