\section*{سوال ۶}

تحلیل روش طراحی ویژگی‌رانه (نسخه سوم) را به دقت مطالعه کنید و تحلیل خود را از این روش براساس موارد زیر بیان کنید.

\section*{مستندات معماری}
شامل تصمیمات، عقلانیت، دیدگاه‌های معماری، راه‌حل‌های جایگزین، بازنمایی و سایر موارد اشاره شده در کتاب.

\section*{نگرانی‌های همه انواع ذی‌نفعان}
شامل کاربر نهایی، مشتری، تیم ایجاد، مدیر پروژه و …

\section*{چگونگی کاربرد مفاهیم، اصول، الگوها و سبک‌های معماری}
بررسی نحوه کاربرد این مفاهیم در روش طراحی ویژگی رانه.

توجه کنید پوشایی و دقت پاسخ شما به عنوان ملاک مهم ارزیابی در این سوال در نظر گرفته می‌شود.

\section*{جواب سوال ۶}

\begin{figure}[H]
	\centering
	\includegraphics{pic10.png}
	\label{fig:label4}
\end{figure}

روش طراحی ویژگی‌رانه (ADD) یک رویکرد چابک برای توسعه معماری نرم‌افزار است که بر اساس ویژگی‌ها متمرکز است. این روش در سال 2004 توسط دو نفر به نام‌های جف باک و ریچارد هافمن معرفی شد و در سال 2016 نسخه سوم آن منتشر شد.

\section{مستندات معماری}

در روش طراحی ویژگی‌رانه، مستندات معماری به صورت مداوم و در طول فرآیند توسعه ایجاد می‌شود. این مستندات شامل موارد زیر هستند:

\begin{itemize}
	\item **دستور کار (Action Plan)**: این سند شامل جزئیات هر گام از فرآیند طراحی ویژگی‌رانه است. این سند باید شامل موارد زیر باشد:
	* اهداف هر گام
	* فعالیت‌های مورد نیاز برای رسیدن به اهداف
	* زمان‌بندی هر فعالیت
	* منابع مورد نیاز برای هر فعالیت
	\item **استانداردها و الگوها (Standards and Patterns)**: این سند شامل استانداردها و الگوهای مورد استفاده در طراحی معماری است. این سند باید شامل موارد زیر باشد:
	* استانداردهای فنی، مانند استانداردهای زبان برنامه‌نویسی، استانداردهای پایگاه داده، و استانداردهای امنیت
	* الگوهای معمارانه، مانند الگوهای سه‌لایه، MVC، و MVVM
	\item **توضیحات معماری (Architecture Description)**: این سند شامل توضیحات کامل از معماری نرم‌افزار است. این سند باید شامل موارد زیر باشد:
	* اجزای نرم‌افزار
	* روابط بین اجزای نرم‌افزار
	* ویژگی‌های نرم‌افزار
	* محدودیت‌های نرم‌افزار
\end{itemize}

این مستندات باید به گونه‌ای تهیه شوند که برای همه ذی‌نفعان قابل فهم باشند.

\section{نگرانی‌های همه انواع ذی‌نفعان}

روش طراحی ویژگی‌رانه بر اهمیت توجه به نگرانی‌های همه انواع ذی‌نفعان تأکید دارد. این ذی‌نفعان شامل موارد زیر هستند:

\begin{itemize}
	\item کاربر نهایی \lr{(End User)} : افرادی که نرم‌افزار را استفاده می‌کنند. نگرانی‌های این افراد شامل موارد زیر است:
	کارایی نرم‌افزار
	
	سهولت استفاده از نرم‌افزار
	
	قابلیت اطمینان نرم‌افزار
	
	\item مشتری \lr{(Customer)} : افرادی که نرم‌افزار را سفارش می‌دهند. نگرانی‌های این افراد شامل موارد زیر است:
	هزینه نرم‌افزار
	
	زمان تحویل نرم‌افزار
	
	الزامات قانونی نرم‌افزار
	
	\item تیم ایجاد \lr{(Development Team)} : افرادی که نرم‌افزار را توسعه می‌دهند. نگرانی‌های این افراد شامل موارد زیر است:
	پیچیدگی نرم‌افزار
	
	قابلیت نگهداری نرم‌افزار
	
	قابلیت توسعه‌پذیری نرم‌افزار
	
	\item مدیر پروژه \lr{(Project Manager)} : افرادی که پروژه توسعه نرم‌افزار را مدیریت می‌کنند. نگرانی‌های این افراد شامل موارد زیر است:
	* ریسک‌های پروژه
	* زمان‌بندی پروژه
	* بودجه پروژه
\end{itemize}

روش طراحی ویژگی‌رانه به ذی‌نفعان کمک می‌کند تا در فرآیند طراحی معماری مشارکت داشته باشند و نگرانی‌های خود را بیان کنند.

\section{چگونگی کاربرد مفاهیم، اصول، الگوها و سبک‌های معماری}

روش طراحی ویژگی‌رانه از مفاهیم، اصول، الگوها و سبک‌های معماری برای ایجاد یک معماری نرم‌افزار کارآمد و پایدار استفاده می‌کند. این مفاهیم، اصول، الگوها و سبک‌ها عبارتند از:

\begin{itemize}
	\item **مفاهیم معماری:** مانند مفاهیم سیستم‌های باز، سیستم‌های توزیع‌شده و سیستم‌های چند لایه.
	\item **اصول معماری:** مانند اصل انتزاع، اصل تفکیک وظایف و اصل وابستگی ضعیف.
	\item **الگوهای معماری:** مانند الگوهای معمارانه مشترک مانند سه‌لایه، MVC و MVVM.

\begin{figure}[H]
	\centering
	\includegraphics{pic11.png}
	\label{fig:label4}
\end{figure}