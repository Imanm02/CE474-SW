\section*{سوال ۳}

تفاوت فعالیت های تحلیل و طراحی سیستم های نرم‌افزاری را توضیح دهید. اطمینان حاصل کنید که در توضیحات خود به موارد زیر بپردازید:

\begin{itemize}
	\item ارتباط آن دو با یک مساله و راه حل آن
	\item اهداف و تمرکز هر یک
	\item سطح انتزاع هر کدام
	\item تقدم و تاخر هر یک از این دو فعالیت
	\item تفاوت مدل‌سازی ذیل هر فعالیت
\end{itemize}

\section*{جواب سوال ۳}

\subsection*{ارتباط با مساله و راه حل}
تحلیل نرم‌افزار به درک مسائل و نیازمندی‌های کاربران می‌پردازد و روی شناسایی و تعریف مشکلات تمرکز دارد. طراحی نرم‌افزار، از سوی دیگر، روی ارائه راه‌حل‌های فنی و ساختار سیستم برای برآورده ساختن این نیازمندی‌ها تمرکز دارد.

\subsection*{اهداف و تمرکز}
هدف از تحلیل نرم‌افزار شناسایی، جمع‌آوری و تعریف نیازمندی‌های کاربر است. در طراحی نرم‌افزار، تمرکز بر روی تعریف معماری، اجزا، رابط‌ها و دیگر جنبه‌های سیستم است.

\subsection*{سطح انتزاع}
تحلیل نرم‌افزار در سطح بالایی از انتزاع عمل می‌کند، به دنبال درک کلی مسائل و نیازمندی‌ها است. طراحی نرم‌افزار به سطح پایین‌تری از انتزاع می‌پردازد و به جزئیات فنی و ساختاری می‌پردازد.

\subsection*{تقدم و تاخر}
تحلیل نرم‌افزار معمولاً قبل از طراحی نرم‌افزار انجام می‌شود. تحلیل به شناسایی نیازمندی‌ها و مسائل می‌پردازد، در حالی که طراحی راه‌حل‌هایی برای این نیازمندی‌ها ارائه می‌دهد.

\subsection*{تفاوت در مدل‌سازی}
مدل‌سازی در تحلیل نرم‌افزار بر روی نمایش نیازمندی‌ها و مسائل تمرکز دارد، مانند دیاگرام‌های حالت و مورد استفاده. در طراحی نرم‌افزار، مدل‌سازی به ساختار و روابط بین اجزای سیستم می‌پردازد، مانند دیاگرام‌های کلاس و توالی.