\section*{سوال ۱}

چرا دستیاران آموزشی درس مهندسی نرم‌افزار به این نتیجه رسیده اند که درس، پروژه یا تمرین عملی نداشته باشد؟
\section*{جواب سوال ۱}

چند علت ممکن است وجود داشته باشد که این‌جا به شرح هر کدام از این علت‌ها و توضیح آن‌ها می‌پردازیم:

\begin{enumerate}
	\item هدف اصلی یادگیری این درس: ممکن است از سمت استاد درس و دستیاران، هدف این درس درک مفاهیم مهندسی نرم‌افزار و افزایش مهارت تحلیل دانشجو باشد. پس شاید پروژه‌ی عملی در این راستا نباشد و وجود تمارین تئوری و مفهومی به این هدف کمک بیشتری کنند. در واقع ممکن است هدف‌گذاری استاد درس و دستیاران به این شکل باشد که در پایان این درس، یادگیری مفاهیم و مهارت تحلیل مهم باشد و نه توانایی پیاده‌سازی، برای همین منطقی‌ست که پروژه یا تمرین عملی در سیاست‌های درس جایی نداشته باشند.
	
	\item زمان کم: فشرده بودن برنامه درسی دانشجویان و زمان کم هر ترم باعث می‌شود که پروژه‌های عملی با کیفیت خوب و کامل قابل اجرا نباشند و هماهنگی بین افراد گروه به خوبی انجام نگیرد. برای همین شاید منطقی‌ست که پروژه‌های عملی که به‌خصوص در این درس، زمان زیادی نیاز دارند برای این‌که همه‌ی افراد در آن درگیر شوند، حذف شود.
	
	\item ساختار برنامه‌ی درسی: گاهی اوقات برنامه‌ی درسی به گونه‌ای طراحی شده است که تمرینات عملی در دروس دیگر یا در سطوح بالاتری از تحصیل تعبیه شده‌اند.
	
	\item فرضیه‌های پیش‌زمینه‌ای: ممکن است دستیاران آموزشی فرض کنند که دانشجویان قبلاً مهارت‌های عملی لازم را در دروس دیگر یا از طریق تجربیات شخصی کسب کرده‌اند، برای همین اصلا یادگیری مبحث جدیدی نمی‌تواند برایشان داشته باشد بخش عملی این درس.
	
	\item ارزیابی و سنجش: ممکن است سیستم ارزیابی دانشگاه تمرکز بیشتری بر روی امتحانات کتبی داشته باشد و بر این باور باشد که تمرینات عملی به سادگی قابل ارزیابی و نمره‌دهی نیستند.
\end{enumerate}

در هر صورت، اینکه آیا تصمیم برای عدم شامل کردن پروژه‌ها و تمرینات عملی در این درس، تصمیم درستی است یا خیر، به شرایط خاص کلاس و اهداف آموزشی تیم دستیاران و استاد درس بستگی دارد. با این‌که طبق بیان کتاب پرسمن، بدون تجربه‌ی عملی، دانشجویان نمی‌توانند دانش نظری را به طور کامل درک کنند و آن را در محیط‌های واقعی به کار گیرند، منطقی‌ست که پس از آگاهی از هدف‌گذاری تیم تدریس، به بررسی این موارد بپردازیم و تحلیل کنیم.

