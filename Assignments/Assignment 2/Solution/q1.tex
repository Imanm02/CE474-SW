\section*{سوال ۱}

۱. معماری یک خانه یا ساختمان را در نظر بگیرید و با معماری نرم‌افزار مقایسه کنید.

۲. رشته‌های معماری ساختمان و معماری نرم‌افزار چه شباهت‌هایی دارند؟ چه تفاوت‌هایی دارند؟

\section*{جواب سوال ۱}

\section*{مفهوم معماری}
\begin{itemize}
	\item معماری ساختمان: طراحی و سازماندهی فضای فیزیکی و ساختاری یک ساختمان.
	\item معماری نرم‌افزار: طراحی و سازماندهی ساختار و اجزای یک سیستم نرم‌افزاری و چگونگی تعامل آن‌ها.
\end{itemize}

\section*{شباهت‌ها}
\begin{itemize}
	\item طراحی و برنامه‌ریزی: هر دو نیاز به طراحی و برنامه‌ریزی دقیق قبل از اجرا دارند.
	\item مدیریت منابع: هر دو در مدیریت منابع مانند زمان، بودجه و نیروی انسانی مشابه هستند.
	\item اصول و الگوها: استفاده از اصول و الگوهای طراحی استاندارد.
	\item تغییر و تحول: هر دو در مواجهه با تغییرات و نیازهای جدید قابل توسعه و انعطاف‌پذیر هستند.
\end{itemize}

\section*{تفاوت‌ها}
\begin{itemize}
	\item مادی در برابر مجازی: معماری ساختمان در فضای فیزیکی است، در حالی که معماری نرم‌افزار در فضای مجازی.
	\item دوام و پایداری: ساختمان‌ها معمولاً برای دوام طولانی‌مدت طراحی می‌شوند، در حالی که نرم‌افزارها ممکن است به طور مداوم به‌روزرسانی و بازنویسی شوند.
	\item تکنولوژی و ابزار: ابزار و تکنولوژی‌های مورد استفاده در هر یک متفاوت است.
	\item پیچیدگی و تغییرپذیری: نرم‌افزارها معمولاً دارای پیچیدگی‌های بیشتری در طراحی و تغییرپذیری سریع‌تر هستند.
\end{itemize}