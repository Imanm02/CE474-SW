\section*{سوال ۵}

در این بخش، پنج مفهوم مدل‌سازی عمده در مهندسی نرم‌افزار - DFD ، UML ، User Story ، CRC Card ، و BPMN - مورد بررسی و مقایسه قرار می‌گیرند.

\section*{چه چیزهایی را مدل می‌کنند}
توضیح دهید که هر یک از این مفاهیم چه جنبه‌هایی از سیستم‌های نرم‌افزاری یا فرآیندهای کسب‌وکار را مدل می‌کنند.

\section*{چگونگی مدل‌سازی توسط آن‌ها}
بیان کنید که هر یک از این ابزارها چگونه مفاهیم را مدل می‌کنند و چه نوع نمایشی از اطلاعات ارائه می‌دهند.

\section*{کاربرد و زمان استفاده}
توضیح دهید که هر یک از این مدل‌ها در چه موقعیت‌ها یا مراحل توسعه نرم‌افزار مورد استفاده قرار می‌گیرند.

\section*{تفاوت سطح انتزاع در مدل‌سازی}
بررسی کنید که چگونه سطح انتزاع در هر یک از این مدل‌ها متفاوت است و تأثیر آن بر کاربردشان چیست.



\section*{جواب سوال ۵}

