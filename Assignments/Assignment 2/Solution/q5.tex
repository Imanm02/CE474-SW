\section*{سوال ۵}

	آیا «متدولوژی ایجاد نرم‌افزار» همان «فرآیند ایجاد نرم‌افزار» است؟ از پاسخ خود با جزئیات و تفصیل دفاع کنید.
	
	در «مدل فرآیند عمومی ایجاد نرم‌افزار» در کتاب پرسمن، پنج «فعالیت چارچوبی» معرفی می‌شود که یکی از آن‌ها مدل‌سازی است.
	
	در این فعالیت به صورت کلی چه اقداماتی انجام می‌شود؟ امروزه در عمل چه زبان مدل‌سازی استاندارد شده است؟
	
	توضیح دهید که خروجی‌های فعالیت مدل‌سازی، چگونه و با چه هدفی در فعالیت بعدی از فرآیند عمومی ایجاد نرم‌افزار مورد استفاده قرار می‌گیرد.
	
	آیا فعالیت مدل‌سازی با اصل ششم از اصول دوازده‌گانه‌ی چابک در تناقض است؟ از پاسخ خود با مثال دفاع کنید.


\section*{جواب سوال ۵}

