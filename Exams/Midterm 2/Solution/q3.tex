\section*{سوال ۲}

سناریویی را در نظر بگیرید که در آن کاربر در حال تعامل با یک برنامه موبایل جدید است که برای مدیریت امور مالی شخصی طراحی شده است. این برنامه به کاربران امکان میدهد هزینه‌ها را پیگیری کنند، بودجه را تنظیم کنند و گزارش‌های مالی را مشاهده کنند. با این حال، کاربران برخی از مشکلات را هنگام استفاده از برنامه گزارش کرده‌اند. بر اساس اصول طراحی
\lr{Bruce Tognazzini}
 ، مشخص کنید کدام اصل(ها) ممکن است در این سناریو نقض شده باشد و دلایل آن را بیان کنید.

مشکلات گزارش شده:
\begin{itemize}
	\item برنامه اقدامات مربوطه را پیشنهاد نمی‌کند یا مراحل بعدی کاربر را پیش بینی نمی‌کند، مانند پیشنهاد تنظیم بودجه بر اساس الگوهای هزینه‌های گذشته.
	\item رابط برنامه با عملکردهای بیش از حد در صفحه اصلی به هم ریخته است، که تمرکز روی یک کار واحد مانند وارد کردن هزینه‌های روزانه را دشوار می‌کند.
	\item همینطور کاربران جدید گزارش کرده‌اند که درک نحوه پیمایش در برنامه و استفاده از ویژگی‌های آن مشکل دارند.
	\item ساختار \lr{Navigation} گیج‌کننده است بطوریکه برخی از عملکردها که در زیر چندین لایه از منوها مدفون شده‌اند و یافتن آنها را سخت می‌کند.
\end{itemize}

\section*{جواب سوال ۲}

بر اساس اصول طراحی \lr{Bruce Tognazzini}، چندین اصل ممکن است در این سناریو نقض شده باشند:

\begin{enumerate}
	\item \textbf{قابلیت پیش‌بینی \lr{(Predictability)}}: برنامه باید توانایی پیش‌بینی نیازهای کاربر و ارائه پیشنهادات مفید را داشته باشد. نبود این ویژگی در برنامه نشان‌دهنده نقض این اصل است. عدم پیشنهاد تنظیم بودجه بر اساس الگوهای هزینه‌های گذشته نمونه‌ای از نقض این اصل است.
	
	\item \textbf{سادگی رابط کاربری \lr{(Simplicity)}}: رابط کاربری باید ساده و قابل فهم باشد. ازدحام عملکردها در صفحه اصلی موجب پیچیدگی و دشواری در استفاده می‌شود که نقض این اصل محسوب می‌شود. این می‌تواند باعث سردرگمی کاربران شود و تجربه کاربری را به شدت کاهش دهد.
	
	\item \textbf{قابلیت پیمایش \lr{(Navigability)}}: کاربران باید بتوانند به راحتی در برنامه حرکت کنند و به ویژگی‌های مختلف دسترسی داشته باشند. گیج‌کننده بودن ساختار \lr{Navigation} نشان‌دهنده نقض این اصل است. این مشکل می‌تواند باعث شود که کاربران در استفاده از برنامه دچار مشکل شوند.
	
	\item \textbf{وضوح و شفافیت \lr{(Clarity)}}: کاربران باید به راحتی بتوانند از عملکرد و نحوه استفاده هر قسمت از برنامه آگاه شوند. عدم وجود شفافیت و وضوح در نحوه پیمایش و استفاده از ویژگی‌ها، مانند عملکردهای پنهان در منوها، نشان‌دهنده نقض این اصل است.
	
	\item \textbf{پاسخگویی \lr{(Responsiveness)}}: برنامه باید به نیازهای کاربران به شکلی سریع و مؤثر پاسخ دهد. در صورتی که برنامه نتواند به سرعت و به طور مناسب به ورودی‌ها و درخواست‌های کاربران پاسخ دهد، این اصل نقض شده است.
\end{enumerate}

منتها حالا به صورت جزئی به بررسی موارد در نظر گرفته نشده در هر کدام از سناریوها اگر بپردازیم، می‌توانیم نتیجه بگیریم که:

\begin{itemize}
	\item \textbf{سناریوی اول}: در این سناریو، اصل \lr{Anticipation} و \lr{Efficiency} مورد توجه قرار نگرفته‌اند. این به معنای عدم پیش‌بینی نیازهای کاربر و ارائه راه‌حل‌های کارآمد است. برای مثال، برنامه می‌توانست با تحلیل هزینه‌های گذشته کاربر، پیشنهاداتی برای مدیریت بهتر بودجه ارائه دهد، که این امر نادیده گرفته شده است.
	
	\item \textbf{سناریوی دوم}: این سناریو مشکلاتی در رابط کاربری دارد که نشان‌دهنده عدم توجه به \lr{Focus}، \lr{Learnability}، و \lr{Readability} است. رابط کاربری باید به گونه‌ای طراحی شود که کاربر بتواند به سرعت و بدون سردرگمی به وظایف خود دست یابد، اما این رابط کاربری پیچیده و دشوار به فهم است.
	
	\item \textbf{سناریوی سوم}: در این سناریو، \lr{Learnability} و \lr{Navigation} به خوبی در نظر گرفته نشده‌اند. کاربران جدید با چالش‌هایی در درک نحوه کار با برنامه و پیمایش در آن روبرو هستند که نشان‌دهنده کمبود شفافیت و سادگی در طراحی است.
	
	\item \textbf{سناریوی چهارم}: اصل \lr{Discoverability} و \lr{Navigation} در این سناریو نادیده گرفته شده‌اند. عملکردهای مهم در زیر لایه‌های متعدد منو مخفی شده‌اند که دسترسی به آن‌ها را دشوار می‌سازد و کاربران را گیج می‌کند.
\end{itemize}