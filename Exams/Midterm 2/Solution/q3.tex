\section*{سوال ۲}

سناریویی را در نظر بگیرید که در آن کاربر در حال تعامل با یک برنامه موبایل جدید است که برای مدیریت امور مالی شخصی طراحی شده است. این برنامه به کاربران امکان میدهد هزینه‌ها را پیگیری کنند، بودجه را تنظیم کنند و گزارش‌های مالی را مشاهده کنند. با این حال، کاربران برخی از مشکلات را هنگام استفاده از برنامه گزارش کرده‌اند. بر اساس اصول طراحی
\lr{Bruce Tognazzini}
 ، مشخص کنید کدام اصل(ها) ممکن است در این سناریو نقض شده باشد و دلایل آن را بیان کنید.

مشکلات گزارش شده:
\begin{itemize}
	\item برنامه اقدامات مربوطه را پیشنهاد نمی‌کند یا مراحل بعدی کاربر را پیش بینی نمی‌کند، مانند پیشنهاد تنظیم بودجه بر اساس الگوهای هزینه‌های گذشته.
	\item رابط برنامه با عملکردهای بیش از حد در صفحه اصلی به هم ریخته است، که تمرکز روی یک کار واحد مانند وارد کردن هزینه‌های روزانه را دشوار می‌کند.
	\item همینطور کاربران جدید گزارش کرده‌اند که درک نحوه پیمایش در برنامه و استفاده از ویژگی‌های آن مشکل دارند.
	\item ساختار \lr{Navigation} گیج‌کننده است بطوریکه برخی از عملکردها که در زیر چندین لایه از منوها مدفون شده‌اند و یافتن آنها را سخت می‌کند.
\end{itemize}

\section*{جواب سوال ۲}

بر اساس اصول طراحی
\lr{Bruce Tognazzini}
 ، چندین اصل ممکن است در این سناریو نقض شده باشند:

\begin{enumerate}
	\item \textbf{قابلیت پیش‌بینی \lr{(Predictability)}}: برنامه باید توانایی پیش‌بینی نیازهای کاربر و ارائه پیشنهادات مفید را داشته باشد. نبود این ویژگی در برنامه نشان‌دهنده نقض این اصل است.
	
	\item \textbf{سادگی رابط کاربری \lr{(Simplicity)}}: رابط کاربری باید ساده و قابل فهم باشد. ازدحام عملکردها در صفحه اصلی موجب پیچیدگی و دشواری در استفاده می‌شود که نقض این اصل محسوب می‌شود.
	
	\item \textbf{قابلیت پیمایش \lr{(Navigability)}}: کاربران باید بتوانند به راحتی در برنامه حرکت کنند و به ویژگی‌های مختلف دسترسی داشته باشند. گیج‌کننده بودن ساختار 
	\lr{Navigation}
	نشان‌دهنده نقض این اصل است.
\end{enumerate}