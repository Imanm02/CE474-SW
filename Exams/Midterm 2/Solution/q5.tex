\section*{سوال ۴}

سناریوهای زیر را در نظر بگیرید. هر سناریو ممکن است اصول طراحی نرم‌افزار شامل اصل

\lr{OCP (Open/Closed Principle)}
، اصل
\lr{LSP (Liskov Substitution Principle)}
، اصل

\lr{DIP (Dependency Inversion Principle)}
، اصل
\lr{ISP (Interface Segregation Principle)}
را نقض کرده باشند و یا دارای
\lr{Coupling}
بالا یا
\lr{Cohesion}
پایین باشند. بررسی نمایید هر یک از این سناریوها چه مشکلی دارند و چرا؟

\begin{enumerate}
	\item یک کلاس \lr{Animal} متدی به نام \lr{makeSound} دارد. کلاس \lr{Dog} و کلاس \lr{Cat} هر دو از \lr{Animal} به ارث می‌برند. نوع جدیدی از حیوانات به نام \lr{Fish} اضافه می‌شود، ولی متد \lr{makeSound} برای آن استفاده نمی‌شود.
	\item یک کلاس رابط کاربری مسئول مدیریت ورودی‌های ماوس، ورودی‌های صفحه کلید، رندر کردن گرافیک و مدیریت پیام‌های شبکه است.
	\item کلاس دسترسی به پایگاه داده یک سیستم نرم‌افزاری از مصرف‌کنندگان می‌خواهد که مدیریت تراکنش، مدیریت اتصال و مدیریت خطا را پیاده‌سازی کنند، حتی اگر فقط به اجرای یک عملیات خواندن ساده نیاز داشته باشند.
	\item یک سیستم پردازش پرداخت به یک درگاه پرداخت خاص وابسته است و هر تغییری که در درگاه پرداخت ایجاد شود، تأثیر مستقیمی بر سیستم پردازش پرداخت خواهد داشت.
\end{enumerate}

\section*{جواب سوال ۴}

در این سناریوها، مشکلات زیر ممکن است وجود داشته باشند:

\begin{enumerate}
	\item \textbf{نقض اصل \lr{LSP} و اصل \lr{OCP}}: اضافه کردن کلاس \lr{Fish} که متد \lr{makeSound} را استفاده نمی‌کند، می‌تواند نقض اصل \lr{Liskov Substitution Principle (LSP)} باشد زیرا کلاس \lr{Fish} نمی‌تواند به درستی جایگزینی برای کلاس \lr{Animal} باشد. همچنین، ممکن است این امر نقض اصل \lr{Open/Closed Principle (OCP)} باشد، زیرا افزودن کلاس جدید نیازمند تغییر در کلاس پایه است.
	
	\item \textbf{نقض اصل \lr{ISP}}: کلاس رابط کاربری با مسئولیت‌های متعدد، مانند مدیریت ورودی‌های ماوس و صفحه کلید، رندر کردن گرافیک و مدیریت پیام‌های شبکه، نقض اصل \lr{Interface Segregation Principle (ISP)} است. این کلاس باید به کلاس‌های کوچکتر با مسئولیت‌های محدودتر تقسیم شود تا هر کلاس تنها مسئولیت‌های مرتبط با خود را بر عهده گیرد.
	
	\item \textbf{\lr{Cohesion} پایین و نقض اصل \lr{DIP}}: کلاس دسترسی به پایگاه داده که از مصرف‌کنندگان می‌خواهد مدیریت تراکنش، اتصال و خطا را پیاده‌سازی کنند، نشان‌دهنده \lr{Cohesion} پایین است. این ممکن است نقض اصل \lr{Dependency Inversion Principle (DIP)} باشد، زیرا کلاس به جای استفاده از انتزاعات، به جزئیات پیاده‌سازی وابسته است.
	
	\item \textbf{\lr{Coupling} بالا}: وابستگی سیستم پردازش پرداخت به یک درگاه پرداخت خاص، نمونه‌ای از \lr{Coupling} بالا است. این وابستگی باعث می‌شود که هر تغییری در درگاه پرداخت، تأثیر مستقیمی بر سیستم پردازش پرداخت داشته باشد، که انعطاف‌پذیری سیستم را محدود می‌کند و به تغییرات احتمالی در آینده حساس می‌شود.
\end{enumerate}

به این ترتیب، می‌توان دریافت که هر یک از این سناریوها نمونه‌هایی از چالش‌های متداول در طراحی نرم‌افزار هستند که اهمیت توجه به اصول مهندسی نرم‌افزار را نشان می‌دهند. این تحلیل به توسعه‌دهندگان کمک می‌کند تا از این اشتباهات در پروژه‌های آینده اجتناب کرده و به ساخت سیستم‌هایی با کیفیت بالاتر و قابلیت نگهداری بهتر بپردازند.