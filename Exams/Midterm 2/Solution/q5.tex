\section*{سوال ۴}

سناریوهای زیر را در نظر بگیرید. هر سناریو ممکن است اصول طراحی نرم‌افزار شامل اصل
\lr{OCP (Open/Closed Principle)}
، اصل
\lr{LSP (Liskov Substitution Principle)}
، اصل
\lr{DIP (Dependency Inversion Principle)}
، اصل
\lr{ISP (Interface Segregation Principle)}
را نقض کرده باشند و یا دارای
\lr{Coupling}
بالا یا
\lr{Cohesion}
پایین باشند. بررسی نمایید هر یک از این سناریوها چه مشکلی دارند و چرا؟

\begin{enumerate}
	\item یک کلاس \lr{Animal} متدی به نام \lr{makeSound} دارد. کلاس \lr{Dog} و کلاس \lr{Cat} هر دو از \lr{Animal} به ارث می‌برند. نوع جدیدی از حیوانات به نام \lr{Fish} اضافه می‌شود، ولی متد \lr{makeSound} برای آن استفاده نمی‌شود.
	\item یک کلاس رابط کاربری مسئول مدیریت ورودی‌های ماوس، ورودی‌های صفحه کلید، رندر کردن گرافیک و مدیریت پیام‌های شبکه است.
	\item کلاس دسترسی به پایگاه داده یک سیستم نرم‌افزاری از مصرف‌کنندگان می‌خواهد که مدیریت تراکنش، مدیریت اتصال و مدیریت خطا را پیاده‌سازی کنند، حتی اگر فقط به اجرای یک عملیات خواندن ساده نیاز داشته باشند.
	\item یک سیستم پردازش پرداخت به یک درگاه پرداخت خاص وابسته است و هر تغییری که در درگاه پرداخت ایجاد شود، تأثیر مستقیمی بر سیستم پردازش پرداخت خواهد داشت.
\end{enumerate}

\section*{جواب سوال ۴}

در این سناریوها، مشکلات زیر ممکن است وجود داشته باشند:

\begin{enumerate}
	\item \textbf{نقض اصل LSP و اصل OCP}: در مورد کلاس \lr{Fish} که متد \lr{makeSound} را استفاده نمی‌کند، این نشان‌دهنده نقض اصل Liskov Substitution (LSP) است زیرا \lr{Fish} نمی‌تواند به درستی جایگزین \lr{Animal} شود. همچنین، این ممکن است نقض اصل Open/Closed (OCP) باشد زیرا برای اضافه کردن \lr{Fish} نیاز به تغییر در کلاس \lr{Animal} است.
	
	\item \textbf{نقض اصل ISP}: کلاس رابط کاربری که مسئولیت‌های زیادی دارد، نقض اصل Interface Segregation (ISP) است. بهتر است که وظایف به کلاس‌های تخصصی‌تر تقسیم شوند تا هر کلاس فقط مسئولیت‌های مرتبط با خود را داشته باشد.
	
	\item \textbf{Cohesion پایین و نقض اصل DIP}: کلاس دسترسی به پایگاه داده با مسئولیت‌های گسترده و اجبار مصرف‌کنندگان برای پیاده‌سازی تمام ویژگی‌ها، نشان‌دهنده Cohesion پایین است. این همچنین ممکن است نقض اصل Dependency Inversion (DIP) باشد، زیرا کلاس به جای استفاده از انتزاعات، مستقیماً به جزئیات پیاده‌سازی وابسته است.
	
	\item \textbf{Coupling بالا}: سیستم پردازش پرداخت که به شدت به یک درگاه پرداخت خاص وابسته است، نمونه‌ای از Coupling بالا است. این وابستگی موجب می‌شود که هر تغییری در درگاه پرداخت تأثیر مستقیمی بر سیستم داشته باشد.
\end{enumerate}