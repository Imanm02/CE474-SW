\section*{سوال ۳}

چه زمانی از
\lr{Component Wrapping}
استفاده می‌کنیم؟ تکنیک‌های مورد استفاده در آن را مختصر توضیح دهید.


\section*{جواب سوال ۳}

\lr{Component Wrapping}
یک تکنیک در مهندسی نرم‌افزار است که برای ادغام کامپوننت‌های موجود در یک سیستم جدید یا برای افزایش سازگاری بین کامپوننت‌های مختلف استفاده می‌شود. از 
\lr{Component Wrapping}
عمدتاً در موقعیت‌های زیر استفاده می‌شود:

\begin{itemize}
	\item \textbf{انطباق با معماری‌های جدید}: زمانی که نیاز است تا کامپوننت‌های موجود در یک معماری جدید بدون تغییر عمده کد اصلی قرار گیرند.
	\item \textbf{پنهان‌سازی پیچیدگی}: برای مخفی کردن پیچیدگی‌های داخلی یک کامپوننت و فراهم کردن یک رابط ساده‌تر برای استفاده‌کنندگان.
	\item \textbf{افزایش قابلیت استفاده مجدد}: امکان استفاده مجدد از کامپوننت‌های قدیمی در محیط‌های جدید با استفاده از رابط‌های جدید.
\end{itemize}

تکنیک‌های مورد استفاده در
\lr{Component Wrapping}
شامل:
\begin{itemize}
	\item \lr{Adapter Pattern}: استفاده از یک واسط یا آداپتور برای تبدیل رابط یک کامپوننت به رابطی دیگر که مورد نیاز است.
	\item \lr{Facade Pattern}: ایجاد یک واسط ساده برای یک سیستم پیچیده، برای ساده‌سازی دسترسی به آن.
\end{itemize}