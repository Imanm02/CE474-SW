\section*{سوال ۳}

چه زمانی از
\lr{Component Wrapping}
استفاده می‌کنیم؟ تکنیک‌های مورد استفاده در آن را مختصر توضیح دهید.


\section*{جواب سوال ۳}

\lr{Component Wrapping} یک تکنیک در مهندسی نرم‌افزار است که برای ادغام کامپوننت‌های موجود در یک سیستم جدید یا برای افزایش سازگاری بین کامپوننت‌های مختلف استفاده می‌شود. کاربردها و تکنیک‌های مورد استفاده در \lr{Component Wrapping} عبارتند از:

\begin{itemize}
	\item \textbf{انطباق با معماری‌های جدید}: برای ادغام کامپوننت‌های قدیمی در معماری‌های جدید بدون نیاز به بازنویسی کد.
	\item \textbf{پنهان‌سازی پیچیدگی}: مخفی کردن جزئیات داخلی کامپوننت و ارائه رابط کاربری ساده‌تر.
	\item \textbf{افزایش قابلیت استفاده مجدد}: استفاده مجدد از کامپوننت‌های موجود در محیط‌های مختلف با اندکی یا بدون تغییر در کد اصلی.
	\item \textbf{تسهیل تعامل بین کامپوننت‌ها}: تسهیل ارتباط بین کامپوننت‌هایی که ممکن است در ابتدا برای کار با یکدیگر طراحی نشده باشند.
\end{itemize}

تکنیک‌های مورد استفاده در \lr{Component Wrapping} شامل:

\begin{itemize}
	\item \textbf{\lr{Adapter Pattern}}: استفاده از یک آداپتور برای تبدیل رابط یک کامپوننت به رابطی دیگر که مناسب سیستم جدید است.
	\item \textbf{\lr{Facade Pattern}}: ایجاد یک واسط ساده برای دسترسی به یک سیستم پیچیده، که می‌تواند چندین کامپوننت پیچیده را پنهان کند.
	\item \textbf{\lr{Proxy Pattern}}: استفاده از یک کامپوننت نماینده برای کنترل دسترسی به کامپوننت اصلی، مفید برای کنترل دسترسی یا افزودن عملکردهای اضافی.
	\item \textbf{\lr{Decorator Pattern}}: اضافه کردن رفتار جدید به یک کامپوننت موجود بدون تغییر در کد اصلی آن کامپوننت.
\end{itemize}

استفاده از این تکنیک‌ها در \lr{Component Wrapping} به توسعه‌دهندگان اجازه می‌دهد تا از مزایای کامپوننت‌های موجود بهره‌مند شوند و در عین حال انعطاف‌پذیری لازم برای ادغام با سیستم‌های جدید را داشته باشند.