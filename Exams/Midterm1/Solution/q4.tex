\section*{سوال ۳}

چرا ممکن است یک سیستم با عمر طولانی به اسناد طراحی بیشتری نیاز داشته باشد؟

\section*{جواب سوال ۳}

\section*{پشتیبانی و نگهداری}
اسناد طراحی کامل و به‌روز به تیم‌های مختلف کمک می‌کنند تا درک بهتری از سیستم داشته باشند، به ویژه در زمان انتقال مسئولیت نگهداری. به خصوص زمان انتقال یک سیستم و ملحقاتش، اسناد طراحی باعث می‌شوند این انتقال مسئولیت، با درک خوبی همراه شود و جزئیات زیادی که ممکن است در نگاه‌های اول دیده نشوند، ثبت شده باشند و توسط تیم جدید، تشخیص داده شوند.

\section*{تغییرات و به‌روزرسانی‌ها}
اسناد طراحی دقیق می‌توانند به ثبت تغییرات و اطمینان از انسجام کلی سیستم در طول زمان کمک کنند. در کنار کیفیت بالای داکیومنتیشن یا همون مستندات، کیفیت بالای به‌روزرسانی‌ها و تغییرات نیز می‌تواند رخ دهد با توجه به اسناد طراحی دقیق.

\section*{کاهش خطای انسانی}
اسناد طراحی کامل به کاهش خطر از دست دادن دانش و تجربه مرتبط با سیستم و حفظ دانش حیاتی کمک می‌کنند. این مورد نیز مشابه مورد اول در زمان انتقال سیستم تاثیرگذار است.

\section*{سازگاری با محیط‌های جدید}
اسناد طراحی به شناسایی بخش‌هایی از سیستم که نیاز به تغییر یا به‌روزرسانی دارند، کمک می‌کنند، به خصوص در زمان انطباق با فناوری‌های جدید.