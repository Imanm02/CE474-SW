\section*{سوال ۲}

توضیح دهید Prototyping در چه مواقعی می‌تواند در پیش‌برد پروژه موثر باشد؟

\section*{جواب سوال ۲}

\section*{شفاف‌سازی نیازمندی‌ها}
Prototyping
به شناسایی و تصحیح نیازمندی‌های کاربران و مشتریان کمک می‌کند، به ویژه زمانی که این نیازمندی‌ها به‌طور کامل شناخته نشده باشند.

\section*{بازخورد سریع}
پروتوتایپ‌ها امکان دریافت بازخورد سریع از کاربران و ذینفعان را فراهم می‌کنند، که این امر به بهبود و تکامل سریع‌تر محصول کمک می‌کند.

\section*{کاهش ریسک}
ایجاد نمونه‌های اولیه به تیم توسعه کمک می‌کند تا ریسک‌های مربوط به فناوری یا طراحی را در مراحل اولیه پروژه شناسایی و رفع کنند.

\section*{ارتباط بهتر با ذینفعان}
پروتوتایپ‌ها به تیم توسعه این امکان را می‌دهند تا ایده‌ها و پیشنهادات خود را به شکل عینی به ذینفعان نشان دهند، که این امر می‌تواند به ارتباط و درک متقابل بهتر کمک کند.

\section*{انعطاف‌پذیری در توسعه}
استفاده از پروتوتایپ‌ها به تیم توسعه اجازه می‌دهد تا انعطاف‌پذیری بیشتری در تغییر جهت یا اصلاح طرح‌ها در طول فرآیند توسعه داشته باشند.
