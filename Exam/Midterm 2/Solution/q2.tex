\section*{سوال ۱}

چه تفاوتی بین الگوهای معماری با 
\lr{style}
های معماری وجود دارد؟ مختصرا شرح دهید.

\section*{جواب سوال ۱}

\section*{الگوهای معماری \lr{(Architectural Patterns)}}
الگوهای معماری، راه‌حل‌های تجربه شده و آزموده‌شده‌ای هستند که برای حل مشکلات معماری خاص در طراحی نرم‌افزار به کار می‌روند. این الگوها شامل دستورالعمل‌ها و رهنمودهایی برای توزیع مسئولیت‌ها در بین اجزای نرم‌افزار هستند. مثال‌هایی از الگوهای معماری شامل معماری سه لایه  \lr{(Three-Tier Architecture)}، مدل-نما-کنترلر \lr{(MVC)}، و میکروسرویس‌ها \lr{(Microservices)} هستند.

\section*{Style های معماری \lr{(Architectural Styles)}}
\lr{Style} های معماری، بیشتر بر روی اصول و مفاهیم کلی در طراحی سیستم‌های نرم‌افزاری تمرکز دارند و کمتر به جزئیات پیاده‌سازی می‌پردازند. آن‌ها چارچوب کلی‌تری برای درک و بیان ساختار یک سیستم فراهم می‌کنند. مثال‌هایی از Style های معماری شامل سرویس‌گرا \lr{(SOA)}، رویداد محور \lr{(Event-Driven)}، و منشوری \lr{(Layered)} هستند.

\textbf{تفاوت‌های کلیدی}:
\begin{enumerate}
	\item \textit{محدوده کاربرد}: الگوهای معماری اغلب با جزئیات بیشتری برای حل مشکلات مشخص در نرم‌افزار تعریف می‌شوند، در حالی که Style های معماری مفاهیم کلی‌تر و چارچوب‌های فکری را ارائه می‌دهند.
	\item \textit{جزئیات و دستورالعمل‌ها}: الگوهای معماری معمولاً دستورالعمل‌های مشخص‌تری برای پیاده‌سازی دارند، در حالی که Style های معماری بیشتر به بیان اصول و مفاهیم پایه‌ای می‌پردازند.
	\item \textit{انعطاف‌پذیری}: Style های معماری اغلب انعطاف‌پذیری بیشتری برای تطبیق با شرایط مختلف دارند، در حالی که الگوهای معماری ممکن است در شرایط خاصی محدودیت‌هایی داشته باشند.
	\item \textit{تمرکز بر کیفیت‌های سیستم}: Style های معماری تمرکز بیشتری بر کیفیت‌های کلی سیستم مانند قابلیت اطمینان، امنیت، و قابلیت استفاده دارند، در حالی که الگوهای معماری اغلب بر حل مسائل فنی و معماری تمرکز دارند.
	\item \textit{تطبیق‌پذیری و مقیاس‌پذیری}: در حالی که الگوهای معماری ممکن است در پیاده‌سازی‌های خاص محدودیت‌هایی داشته باشند، Style های معماری اغلب اجازه می‌دهند که سیستم‌ها با تغییرات تکنولوژیکی یا نیازهای تجاری به راحتی تطبیق یابند.
\end{enumerate}