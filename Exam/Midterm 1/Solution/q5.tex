\section*{سوال ۴}

\section*{\lr{.A}}
 چرا باید برای ایجاد یک نرم‌افزار براساس یک مدل پیش برویم و در طول پروژه پایبند به آن مدل باشیم؟

\section*{\lr{.B}}
یک تیم مهندس نرم‌افزار برای پروژه‌ای در یک شرکت نفت بزرگ دعوت شده است. این شرکت چندین دپارتمان دارد و تیم مهندسی نرم‌افزار با دپارتمان مدیریت اطلاعات (MIS) تعامل می‌کند. سیستم MIS این شرکت قدیمی (Legacy) است و هدف، انتقال داده‌ها به یک سیستم جدید (مهاجرت داده) است. فرآیندها، قراردادهای قانونی و معیارهای پذیرش این شرکت بسیار خاص و حساس هستند. به نظر شما چه مدل ایجاد نرم‌افزاری برای راه‌اندازی این سیستم انتقال داده را تیم مهندس نرم‌افزار انتخاب خواهد کرد؟ نام مدل و علت اصلی انتخاب آن کافی است.

\section*{\lr{.C}}
مهم‌ترین مشکلات مدل‌های سنتی (مثل مدل آبشاری) نسبت به مدل‌های چابک، چیست؟ ()اشاره به ۳ مورد و توضیح کامل آنها کفایت می‌کند.)

\section*{جواب سوال ۴}

\section*{\lr{.A} اهمیت استفاده از یک مدل در ایجاد نرم‌افزار}

\begin{itemize}
	\item \textbf{ساختار و راهنمایی:}
	مدل‌ها چارچوب و راهنمایی لازم را برای توسعه نرم‌افزار فراهم می‌کنند، کمک به تمرکز تیم بر اهداف و مراحل مشخص. در واقع نظم تیم با استفاده از یک مدل، خیلی دقیق برنامه‌ریزی می‌شود.
	\item \textbf{مدیریت پیچیدگی:}
	استفاده از یک مدل پیچیدگی‌ها را مدیریت می‌کند و اطمینان می‌دهد که تمام جنبه‌های پروژه پوشش داده شوند. بعضا برای ایجاد نظم در تیم‌ها، از روش‌های مختلفی استفاده می‌شود ولی در پیچیدگی‌ها، مدیریت سخت می‌شود و بعضی بخش‌ها پوشش داده نمی‌شوند. برای همین بهتر از از یک مدل برای مدیریت بخش‌های پیچیده‌ی یک پروژه استفاده شود.
	\item \textbf{کنترل کیفیت:}
	پایبندی به مدل امکان بررسی و ارزیابی مرحله به مرحله پروژه را می‌دهد، که برای حفظ کیفیت نرم‌افزار ضروری است. همانند مورد قبل، کیفیت در بعضی بخش‌های پروژه ممکن است حفظ نشود در مدیریت عادی ولی با استفاده از مدل‌ها، این موضوع نیز پوشش داده می‌شود.
	\item \textbf{پیش‌بینی و برنامه‌ریزی:}
	مدل‌ها به تیم توسعه امکان پیش‌بینی و برنامه‌ریزی مناسب پیشرفت پروژه را می‌دهند.
	\item \textbf{همکاری و ارتباطات:}
	مدل‌ها زبان مشترکی برای ارتباط بین اعضای تیم و ذینفعان فراهم می‌کنند.
	\item \textbf{مستندسازی و توسعه مجدد:}
	مدل‌ها به مستندسازی فرایندها و تصمیمات کمک کرده و برای تحلیل و توسعه مجدد نرم‌افزار در آینده مهم هستند.
\end{itemize}

\section*{\lr{.B} مدل ایجاد نرم‌افزار برای شرکت نفتی}
\textbf{مدل انتخابی:}
مدل
\lr{V}

\textbf{دلیل انتخاب:}

\begin{itemize}

	\begin{enumerate}
		\item \textbf{تاکید بر آزمون و اعتبارسنجی:}
	مدل V به خصوص برای پروژه‌هایی که نیاز به تایید دقیق و مستمر دارند، مناسب است. این مهم است زیرا در پروژه‌های مربوط به داده‌های حساس مانند مهاجرت داده‌ها در یک شرکت نفتی، اطمینان از دقت و امنیت داده‌ها در هر مرحله حیاتی است.
	
		\item \textbf{مدیریت ریسک در مهاجرت داده:}
	 مهاجرت داده‌ها از یک سیستم قدیمی به سیستم جدید می‌تواند پیچیده باشد. مدل V با تاکید بر آزمون‌ها در هر مرحله از فرآیند توسعه، این امکان را فراهم می‌کند که اشکالات و مسائل احتمالی زودتر شناسایی و برطرف شوند.
	 
		\item \textbf{پاسخ‌گویی به نیازمندی‌های خاص:}
	توجه به فرآیندها و قراردادهای قانونی خاص و حساسیت‌های شرکت نفتی نشان‌دهنده نیاز به یک رویکرد دقیق و ساختارمند است. مدل V با ارائه یک چارچوب مرحله‌به‌مرحله، این نیازها را برآورده می‌کند.
		\item \textbf{ساختارمند و منظم:} این مدل با فرض اینکه نیازمندی‌ها به درستی تعریف شده‌اند، یک رویکرد منظم و ساختارمند ارائه می‌دهد که برای پروژه‌های با اهداف و فرآیندهای واضح مفید است.
	\end{enumerate}
	
	با این حال، اگر فرآیند انتقال داده دارای ابهامات زیادی باشد و نیاز به تغییرات مکرر و ارزیابی‌های مداوم ریسک داشته باشد، مدل Spiral گزینه بهتری خواهد بود. مدل Spiral به خصوص برای پروژه‌های پیچیده و نوآورانه که نیازمند انعطاف‌پذیری و ارزیابی مداوم ریسک هستند، مناسب است. این مدل اجازه می‌دهد تا تیم مهندسی در طول پروژه به طور مداوم فرآیندها و نتایج را ارزیابی کرده و در صورت لزوم تغییرات را اعمال کند.
	
\end{itemize}

\section*{\lr{.C} مشکلات مدل‌های سنتی (مانند مدل آبشاری) نسبت به مدل‌های چابک}
\begin{itemize}
	\item \textbf{انعطاف‌پذیری کم:}
	مدل‌های سنتی، معمولاً انعطاف‌پذیری کمی دارند. این بدان معناست که تغییرات در الزامات پروژه، می‌تواند منجر به مشکلات زیادی شود. مدل‌های چابک، انعطاف‌پذیری بیشتری دارند و به تیم توسعه نرم‌افزار اجازه می‌دهند تا تغییرات در الزامات پروژه را به سرعت شناسایی و اعمال کنند.
	
	\item \textbf{کمبود تعامل با مشتری:}
	مدل‌های سنتی، معمولاً تعامل کمی با مشتری دارند. این امر می‌تواند منجر به عدم رضایت مشتری شود. مدل‌های چابک، بر تعامل نزدیک با مشتری تأکید دارند. این امر به تیم توسعه نرم‌افزار کمک می‌کند تا نیازهای مشتری را به طور دقیق و کامل شناسایی کنند و محصولاتی را تولید کنند که رضایت مشتری را جلب کند.
	
	\item \textbf{تأخیر در بازخورد:}
	در مدل‌های سنتی، بازخورد کاربران و ذینفعان معمولاً در مراحل پایانی پروژه جمع‌آوری می‌شود، که می‌تواند منجر به تأخیر در شناسایی و حل مشکلات شود.
	
	\item \textbf{ریسک بالا و هزینه‌های تغییر:}
	به دلیل تأخیر در دریافت بازخورد و انعطاف‌پذیری کم، ریسک شکست پروژه‌ها و هزینه‌های ایجاد تغییرات افزایش می‌یابد.
	
	\item \textbf{تمرکز بر مستندات:}
	مدل‌های سنتی، معمولاً بر مستندات تأکید دارند. این امر می‌تواند منجر به صرف زمان و هزینه زیاد شود. مدل‌های چابک، بر تحویل ارزش به مشتری در فواصل زمانی کوتاه تأکید دارند. این امر می‌تواند منجر به صرف زمان و هزینه کمتر شود.
	
\end{itemize}