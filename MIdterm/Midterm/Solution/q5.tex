\section*{سوال ۴}

\section*{\lr{.A}}
 چرا باید برای ایجاد یک نرم‌افزار براساس یک مدل پیش برویم و در طول پروژه پایبند به آن مدل باشیم؟

\section*{\lr{.B}}
یک تیم مهندس نرم‌افزار برای پروژه‌ای در یک شرکت نفت بزرگ دعوت شده است. این شرکت چندین دپارتمان دارد و تیم مهندسی نرم‌افزار با دپارتمان مدیریت اطلاعات (MIS) تعامل می‌کند. سیستم MIS این شرکت قدیمی (Legacy) است و هدف، انتقال داده‌ها به یک سیستم جدید (مهاجرت داده) است. فرآیندها، قراردادهای قانونی و معیارهای پذیرش این شرکت بسیار خاص و حساس هستند. به نظر شما چه مدل ایجاد نرم‌افزاری برای راه‌اندازی این سیستم انتقال داده را تیم مهندس نرم‌افزار انتخاب خواهد کرد؟ نام مدل و علت اصلی انتخاب آن کافی است.

\section*{\lr{.C}}
مهم‌ترین مشکلات مدل‌های سنتی (مثل مدل آبشاری) نسبت به مدل‌های چابک، چیست؟ ()اشاره به ۳ مورد و توضیح کامل آنها کفایت می‌کند.)

\section*{جواب سوال ۴}

\section*{\lr{.A} اهمیت استفاده از یک مدل در ایجاد نرم‌افزار}

\begin{itemize}
	\item \textbf{ساختار و راهنمایی:}
	مدل‌ها چارچوب و راهنمایی لازم را برای توسعه نرم‌افزار فراهم می‌کنند، کمک به تمرکز تیم بر اهداف و مراحل مشخص. در واقع نظم تیم با استفاده از یک مدل، خیلی دقیق برنامه‌ریزی می‌شود.
	\item \textbf{مدیریت پیچیدگی:}
	استفاده از یک مدل پیچیدگی‌ها را مدیریت می‌کند و اطمینان می‌دهد که تمام جنبه‌های پروژه پوشش داده شوند. بعضا برای ایجاد نظم در تیم‌ها، از روش‌های مختلفی استفاده می‌شود ولی در پیچیدگی‌ها، مدیریت سخت می‌شود و بعضی بخش‌ها پوشش داده نمی‌شوند. برای همین بهتر از از یک مدل برای مدیریت بخش‌های پیچیده‌ی یک پروژه استفاده شود.
	\item \textbf{کنترل کیفیت:}
	پایبندی به مدل امکان بررسی و ارزیابی مرحله به مرحله پروژه را می‌دهد، که برای حفظ کیفیت نرم‌افزار ضروری است. همانند مورد قبل، کیفیت در بعضی بخش‌های پروژه ممکن است حفظ نشود در مدیریت عادی ولی با استفاده از مدل‌ها، این موضوع نیز پوشش داده می‌شود.
	\item \textbf{پیش‌بینی و برنامه‌ریزی:}
	مدل‌ها به تیم توسعه امکان پیش‌بینی و برنامه‌ریزی مناسب پیشرفت پروژه را می‌دهند.
	\item \textbf{همکاری و ارتباطات:}
	مدل‌ها زبان مشترکی برای ارتباط بین اعضای تیم و ذینفعان فراهم می‌کنند.
	\item \textbf{مستندسازی و توسعه مجدد:}
	مدل‌ها به مستندسازی فرایندها و تصمیمات کمک کرده و برای تحلیل و توسعه مجدد نرم‌افزار در آینده مهم هستند.
\end{itemize}

\section*{\lr{.B} مدل ایجاد نرم‌افزار برای شرکت نفتی}
\textbf{مدل انتخابی:}
مدل ترکیبی
\lr{(Hybrid Model)}
که ترکیبی از مدل آبشاری و چابک است.

\textbf{دلیل انتخاب:}

\begin{itemize}
	\item \textbf{پیچیدگی و حساسیت بالا:}
	 با توجه به پیچیدگی و حساسیت‌های موجود در فرآیندها و داده‌های شرکت نفتی، مدل ترکیبی اجازه می‌دهد تا بخش‌های حساس و پیچیده با دقت بالا و با رویکرد آبشاری پیاده‌سازی شوند.
	\item \textbf{انعطاف‌پذیری:}
	در بخش‌های کمتر حساس، استفاده از رویکردهای چابک به تیم اجازه می‌دهد تا به سرعت به تغییرات پاسخ دهد.
\end{itemize}

\section*{\lr{.C} مشکلات مدل‌های سنتی (مانند مدل آبشاری) نسبت به مدل‌های چابک}
\begin{itemize}
	\item \textbf{انعطاف‌پذیری کم:}
	مدل‌های سنتی، معمولاً انعطاف‌پذیری کمی دارند. این بدان معناست که تغییرات در الزامات پروژه، می‌تواند منجر به مشکلات زیادی شود. مدل‌های چابک، انعطاف‌پذیری بیشتری دارند و به تیم توسعه نرم‌افزار اجازه می‌دهند تا تغییرات در الزامات پروژه را به سرعت شناسایی و اعمال کنند.
	
	\item \textbf{تأخیر در بازخورد:}
	در مدل‌های سنتی، بازخورد کاربران و ذینفعان معمولاً در مراحل پایانی پروژه جمع‌آوری می‌شود، که می‌تواند منجر به تأخیر در شناسایی و حل مشکلات شود.
	\item \textbf{ریسک بالا و هزینه‌های تغییر:}
	به دلیل تأخیر در دریافت بازخورد و انعطاف‌پذیری کم، ریسک شکست پروژه‌ها و هزینه‌های ایجاد تغییرات افزایش می‌یابد.
\end{itemize}