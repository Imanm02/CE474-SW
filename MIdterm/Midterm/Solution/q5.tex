\section*{سوال ۴}

\section*{\lr{.A}}
 چرا باید برای ایجاد یک نرم‌افزار براساس یک مدل پیش برویم و در طول پروژه پایبند به آن مدل باشیم؟

\section*{\lr{.B}}
یک تیم مهندس نرم‌افزار برای پروژه‌ای در یک شرکت نفت بزرگ دعوت شده است. این شرکت چندین دپارتمان دارد و تیم مهندسی نرم‌افزار با دپارتمان مدیریت اطلاعات (MIS) تعامل می‌کند. سیستم MIS این شرکت قدیمی (Legacy) است و هدف، انتقال داده‌ها به یک سیستم جدید (مهاجرت داده) است. فرآیندها، قراردادهای قانونی و معیارهای پذیرش این شرکت بسیار خاص و حساس هستند. به نظر شما چه مدل ایجاد نرم‌افزاری برای راه‌اندازی این سیستم انتقال داده را تیم مهندس نرم‌افزار انتخاب خواهد کرد؟ نام مدل و علت اصلی انتخاب آن کافی است.

\section*{\lr{.C}}
مهم‌ترین مشکلات مدل‌های سنتی (مثل مدل آبشاری) نسبت به مدل‌های چابک، چیست؟ ()اشاره به ۳ مورد و توضیح کامل آنها کفایت می‌کند.)

\section*{جواب سوال ۴}

\section*{\lr{.A} اهمیت استفاده از یک مدل در ایجاد نرم‌افزار}
\textbf{ساختار و نظم:}
مدل‌ها ساختار و نظمی به فرآیند توسعه نرم‌افزار می‌دهند که باعث می‌شود پروژه قابل پیش‌بینی و مدیریت‌پذیر باشد.

\textbf{تعریف وظایف و مسئولیت‌ها:}
مدل‌ها وظایف و مسئولیت‌های اعضای تیم توسعه را مشخص می‌کنند.

\textbf{کاهش ریسک:}
استفاده از مدل‌ها به شناسایی و مدیریت ریسک‌ها در مراحل اولیه پروژه کمک می‌کند.

\section*{\lr{.B} مدل ایجاد نرم‌افزار برای شرکت نفتی}
\textbf{مدل انتخابی:}
مدل ترکیبی
\lr{(Hybrid Model)}
که ترکیبی از مدل آبشاری و چابک است.

\textbf{دلیل انتخاب:}

\begin{itemize}
	\item \textbf{پیچیدگی و حساسیت بالا:} با توجه به پیچیدگی و حساسیت‌های موجود در فرآیندها و داده‌های شرکت نفتی، مدل ترکیبی اجازه می‌دهد تا بخش‌های حساس و پیچیده با دقت بالا و با رویکرد آبشاری پیاده‌سازی شوند.
	\item \textbf{انعطاف‌پذیری:} در بخش‌های کمتر حساس، استفاده از رویکردهای چابک به تیم اجازه می‌دهد تا به سرعت به تغییرات پاسخ دهد.
\end{itemize}

\section*{\lr{.C} مشکلات مدل‌های سنتی (مانند مدل آبشاری) نسبت به مدل‌های چابک}
\begin{itemize}
	\item \textbf{انعطاف‌پذیری کم:} مدل‌های سنتی اغلب انعطاف‌پذیری کمتری در برابر تغییرات دارند و تغییرات اساسی در مراحل پایانی پروژه دشوار است.
	\item \textbf{تأخیر در بازخورد:} در مدل‌های سنتی، بازخورد کاربران و ذینفعان معمولاً در مراحل پایانی پروژه جمع‌آوری می‌شود، که می‌تواند منجر به تأخیر در شناسایی و حل مشکلات شود.
	\item \textbf{ریسک بالا و هزینه‌های تغییر:} به دلیل تأخیر در دریافت بازخورد و انعطاف‌پذیری کم، ریسک شکست پروژه‌ها و هزینه‌های ایجاد تغییرات افزایش می‌یابد.
\end{itemize}