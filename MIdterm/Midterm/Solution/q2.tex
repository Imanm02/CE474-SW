\section*{سوال ۲}

تفاوت اساسی بین نرم‌افزار و سخت‌افزار با توجه به زمان چیست؟

\section*{جواب سوال ۲}


\section{تغییرپذیری}
نرم‌افزارها به راحتی قابل تغییر و به‌روزرسانی هستند، در حالی که سخت‌افزارها برای تغییر نیاز به تعویض فیزیکی یا ارتقا دارند. با گذشت زمان، این انعطاف‌پذیری نرم‌افزار امکان پاسخ‌گویی به نیازهای جدید را فراهم می‌کند.

\section{عمر مفید}
سخت‌افزارها دارای یک عمر مفید فیزیکی هستند و با گذشت زمان فرسوده می‌شوند. در مقابل، نرم‌افزار فرسودگی فیزیکی ندارد اما ممکن است به دلیل تغییرات فناوری یا نیازهای کاربری، منسوخ شود.

\section{هزینه‌های به‌روزرسانی و نگهداری}
در طول زمان، هزینه‌های نگهداری و به‌روزرسانی نرم‌افزار می‌تواند بسیار بیشتر از هزینه‌های اولیه توسعه آن باشد. در حالی که هزینه‌های سخت‌افزار بیشتر به خرید و نصب اولیه محدود می‌شود.

\section{وابستگی به فناوری}
نرم‌افزارها معمولاً به سرعت تحت تأثیر تغییرات فناوری قرار می‌گیرند. اما سخت‌افزار ممکن است برای مدت زمان طولانی‌تری قابل استفاده باقی بماند، حتی اگر فناوری پیشرفت کند.

\section{مقیاس‌پذیری}
نرم‌افزارها معمولاً به راحتی قابل مقیاس‌پذیری هستند، به این معنی که می‌توان آن‌ها را برای پاسخگویی به نیازهای در حال تغییر تنظیم کرد. سخت‌افزار اغلب نیاز به ارتقا یا تعویض دارد تا بتواند با نیازهای مقیاس بزرگ‌تر سازگار شود.