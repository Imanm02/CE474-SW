\section*{سوال ۵}

\section*{با در نظر گرفتن رویکرد چابک به سوالات زیر پاسخ دهید:}
\subsection*{\lr{.A} در اکثر پروژه‌های نرم‌افزاری پیش‌بینی موارد زیر سخت است:}
\begin{itemize}
	\item اینکه کدام نیازمندی‌های مشتری تغییر خواهند کرد و کدام نیازمندی‌ها ثابت خواهند بود؟
	\item اینکه به چه میزان طراحی پیش از پیاده‌سازی احتیاج داریم؟
	\item و چه مقدار زمان از نظر برنامه‌ریزی برای تحلیل و طراحی، پیاده‌سازی و تست محصول نیاز خواهد بود؟
\end{itemize}
فرآیندهای چابک چگونه در جهت رفع این شرایط‌های نیاز به پیش‌بینی پاسخ می‌دهند؟

\subsection*{\lr{.B}}
 اگر برای سیستم‌های بزرگ و با عمر طولانی که توسط یک شرکت نرم‌افزاری برای مشتریان خارجی توسعه داده می‌شوند، از رویکرد چابک استفاده شود، چه مشکلاتی ممکن است بوجود آید؟. ۳ مورد از مشکلات ممکن را ذکر کنید.

\subsection*{\lr{.C}}
 فکر می‌کنید مدل‌های چابک خود چه مشکلاتی داشته باشند؟ (حداقل ۴ مورد)

\section*{جواب سوال ۵}

\section*{\lr{.A}}
در فرآیندهای چابک، به جای تلاش برای پیش‌بینی دقیق تمام نیازمندی‌ها و طراحی‌ها از ابتدا، تمرکز بر توسعه تدریجی و تکراری محصول است. این رویکرد اجازه می‌دهد تا تیم‌ها به سرعت و با انعطاف‌پذیری بالا به تغییرات نیازمندی‌ها واکنش نشان دهند. زمان‌بندی برای تحلیل، طراحی، پیاده‌سازی و تست نیز به صورت انعطاف‌پذیر در چرخه‌های کوتاه مدت مدیریت می‌شود.

فرآیندهای چابک، بر تعامل نزدیک با مشتری تأکید دارند. این امر به تیم توسعه نرم‌افزار کمک می‌کند تا نیازمندی‌های مشتری را به طور دقیق و کامل شناسایی کنند و تغییرات را به سرعت شناسایی و اعمال کنند.

فرآیندهای چابک، بر آزمایش مداوم تأکید دارند. این امر به تیم توسعه نرم‌افزار کمک می‌کند تا کیفیت محصول را در طول توسعه تضمین کنند و از تغییرات غیرمنتظره در نیازمندی‌های مشتری جلوگیری کنند.

فرآیندهای چابک، انعطاف‌پذیر هستند و اجازه می‌دهند تا تغییرات در نیازمندی‌ها، طراحی و برنامه‌ریزی پروژه به سرعت اعمال شوند. این امر به تیم توسعه نرم‌افزار کمک می‌کند تا به تغییرات محیطی و نیازهای مشتری پاسخ دهند.

فرض کنید یک شرکت نرم‌افزاری برای یک مشتری جدید، یک سیستم اتوماسیون فروش توسعه می‌دهد. مشتری نیازهای خود را به صورت کلی به تیم توسعه نرم‌افزار ارائه می‌دهد. تیم توسعه نرم‌افزار با تعامل نزدیک با مشتری، نیازهای مشتری را به صورت دقیق‌تر شناسایی می‌کند. سپس، محصول را در فواصل زمانی کوتاه تحویل می‌دهد تا مشتری بتواند آن را بررسی کند و تغییرات را درخواست کند.

در این مثال، فرآیندهای چابک به تیم توسعه نرم‌افزار کمک می‌کند تا به تغییرات در نیازمندی‌های مشتری پاسخ دهند. به عنوان مثال، اگر مشتری نیاز به اضافه کردن یک ویژگی جدید به سیستم را داشته باشد، تیم توسعه نرم‌افزار می‌تواند این ویژگی را در نسخه بعدی محصول اعمال کند.

در نتیجه فرآیندهای چابک، با توجه به چالش‌های پیش‌بینی در پروژه‌های نرم‌افزاری، از رویکردهای مختلفی استفاده می‌کنند. این رویکردها به تیم‌های توسعه نرم‌افزار کمک می‌کند تا تغییرات در نیازمندی‌های مشتری را به سرعت شناسایی و اعمال کنند و محصولی با کیفیت بالا تحویل دهند.

\section*{\lr{.B}}
برای سیستم‌های بزرگ و با عمر طولانی، استفاده از رویکرد چابک ممکن است مشکلاتی از قبیل:
\begin{itemize}
	\item دشواری در مدیریت و هماهنگی بین تیم‌های بزرگ و پراکنده.
	\item چالش‌های مربوط به برقراری ارتباط مؤثر با مشتریان بین‌المللی.
	\item مسائل مربوط به مقیاس‌پذیری و ادغام مداوم تغییرات در یک سیستم بزرگ.
\end{itemize}

سیستم‌های بزرگ و با عمر طولانی، معمولاً توسط تیم‌های بزرگ و پراکنده توسعه داده می‌شوند. این تیم‌ها ممکن است در مکان‌های مختلف قرار داشته باشند و از فرهنگ‌ها و زبان‌های مختلف باشند. مدیریت و هماهنگی بین این تیم‌ها، در رویکرد چابک که بر تعامل نزدیک با مشتری تأکید دارد، می‌تواند دشوار باشد.

برای حل این مشکل، تیم‌های توسعه نرم‌افزار باید از ابزارها و تکنیک‌های ارتباطی و همکاری موثر استفاده کنند. همچنین، باید یک برنامه‌ریزی دقیق برای مدیریت و هماهنگی بین تیم‌ها داشته باشند.

اگر سیستم توسط یک شرکت نرم‌افزاری برای مشتریان خارجی توسعه داده شود، تیم توسعه نرم‌افزار باید بتواند با مشتریان بین‌المللی به طور موثر ارتباط برقرار کند. این امر می‌تواند چالش‌هایی را ایجاد کند، به خصوص اگر مشتریان از زبان‌ها و فرهنگ‌های مختلف باشند.

برای حل این مشکل، تیم توسعه نرم‌افزار باید مهارت‌های ارتباطی بین‌المللی داشته باشد. همچنین، باید از ابزارها و تکنیک‌های ارتباطی موثر استفاده کند.

سیستم‌های بزرگ و با عمر طولانی، معمولاً پیچیده هستند و نیاز به نگهداری و پشتیبانی مداوم دارند. در رویکرد چابک، تغییرات در سیستم به صورت مداوم اعمال می‌شوند. این امر می‌تواند چالش‌هایی را در زمینه مقیاس‌پذیری و ادغام تغییرات ایجاد کند.

برای حل این مشکل، تیم توسعه نرم‌افزار باید از ابزارها و تکنیک‌های مناسب برای مدیریت تغییرات استفاده کند. همچنین، باید یک برنامه‌ریزی دقیق برای مدیریت مقیاس‌پذیری سیستم داشته باشد.

البته، استفاده از رویکرد چابک برای سیستم‌های بزرگ و با عمر طولانی، مزایای زیادی نیز دارد. به عنوان مثال، چابکی می‌تواند به تیم‌های توسعه نرم‌افزار کمک کند تا به تغییرات نیازهای مشتری به سرعت پاسخ دهند.

در نهایت، انتخاب رویکرد مناسب برای توسعه یک سیستم نرم‌افزاری، به عوامل مختلفی مانند اندازه سیستم، پیچیدگی سیستم، و الزامات مشتری بستگی دارد.

\section*{\lr{.C}}

مدل‌های چابک، به دلیل انعطاف‌پذیری و تمرکز بر تحویل ارزش به مشتری، مزایای زیادی نسبت به مدل‌های سنتی دارند. با این حال، این مدل‌ها نیز می‌توانند مشکلاتی داشته باشند. در ادامه، چهار مورد از مشکلات احتمالی مدل‌های چابک را بررسی می‌کنیم:

مدل‌های چابک خود می‌توانند مشکلاتی داشته باشند از جمله:
\begin{itemize}
	\item نیاز به ارتباط و همکاری مداوم و نزدیک با مشتری.
	\item دشواری در پیش‌بینی هزینه‌ها و زمان‌بندی‌های بلندمدت.
	\item خطر انحراف از اهداف اصلی در صورت عدم وجود برنامه‌ریزی دقیق.
	\item ممکن است کیفیت کد در اثر تغییرات مکرر کاهش یابد.
\end{itemize}