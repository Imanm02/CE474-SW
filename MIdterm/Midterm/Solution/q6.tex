\section*{سوال ۵}

\section*{با در نظر گرفتن رویکرد چابک به سوالات زیر پاسخ دهید:}
\subsection*{\lr{.A} در اکثر پروژه‌های نرم‌افزاری پیش‌بینی موارد زیر سخت است:}
\begin{itemize}
	\item اینکه کدام نیازمندی‌های مشتری تغییر خواهند کرد و کدام نیازمندی‌ها ثابت خواهند بود؟
	\item اینکه به چه میزان طراحی پیش از پیاده‌سازی احتیاج داریم؟
	\item و چه مقدار زمان از نظر برنامه‌ریزی برای تحلیل و طراحی، پیاده‌سازی و تست محصول نیاز خواهد بود؟
\end{itemize}
فرآیندهای چابک چگونه در جهت رفع این شرایطهای نیاز به پیش‌بینی پاسخ می‌دهند؟

\subsection*{\lr{.B}}
 اگر برای سیستم‌های بزرگ و با عمر طولانی که توسط یک شرکت نرم‌افزاری برای مشتریان خارجی توسعه داده می‌شوند، از رویکرد چابک استفاده شود، چه مشکلاتی ممکن است بوجود آید؟. ۳ مورد از مشکلات ممکن را ذکر کنید.

\subsection*{\lr{.C}}
 فکر می‌کنید مدل‌های چابک خود چه مشکلاتی داشته باشند؟ (حداقل ۴ مورد)

\section*{جواب سوال ۵}

\section*{\lr{.A}}
در فرآیندهای چابک، به جای تلاش برای پیش‌بینی دقیق تمام نیازمندی‌ها و طراحی‌ها از ابتدا، تمرکز بر توسعه تدریجی و تکراری محصول است. این رویکرد اجازه می‌دهد تا تیم‌ها به سرعت و با انعطاف‌پذیری بالا به تغییرات نیازمندی‌ها واکنش نشان دهند. زمان‌بندی برای تحلیل، طراحی، پیاده‌سازی و تست نیز به صورت انعطاف‌پذیر در چرخه‌های کوتاه مدت مدیریت می‌شود.

\section*{\lr{.B}}
برای سیستم‌های بزرگ و با عمر طولانی، استفاده از رویکرد چابک ممکن است مشکلاتی از قبیل:
\begin{itemize}
	\item دشواری در مدیریت و هماهنگی بین تیم‌های بزرگ و پراکنده.
	\item چالش‌های مربوط به برقراری ارتباط مؤثر با مشتریان بین‌المللی.
	\item مسائل مربوط به مقیاس‌پذیری و ادغام مداوم تغییرات در یک سیستم بزرگ.
\end{itemize}

\section*{\lr{.C}}
مدل‌های چابک خود می‌توانند مشکلاتی داشته باشند از جمله:
\begin{itemize}
	\item نیاز به ارتباط و همکاری مداوم و نزدیک با مشتری.
	\item دشواری در پیش‌بینی هزینه‌ها و زمان‌بندی‌های بلندمدت.
	\item خطر انحراف از اهداف اصلی در صورت عدم وجود برنامه‌ریزی دقیق.
	\item ممکن است کیفیت کد در اثر تغییرات مکرر کاهش یابد.
\end{itemize}